%%%%%%%%%%%%%%%%%%%%%%%%%%%%%%%%%%%%%%%%%%%%%%%%%%%%%%%%%%%%%%%%%%%%%%%%%%%
%
% 解析学演習
%
% 黒木 玄 (東北大学理学部数学教室, kuroki@math.tohoku.ac.jp)
%
% この文書は 1995年度前期 解析学序論I で数学系一年生に渡したものです.
%
% 日本語 AMS LaTeX でコンパイルしてください.
%
%%%%%%%%%%%%%%%%%%%%%%%%%%%%%%%%%%%%%%%%%%%%%%%%%%%%%%%%%%%%%%%%%%%%%%%%%%%

%\ifx\gtfam\undefined
% \documentstyle[amstex,amssymb,12pt,enshu]{j-article} % NTT
%\documentstyle[amstex,amssymb,showkeys,12pt,enshu]{j-article}  % NTT
%\else
% \documentstyle[amstex,amssymb,12pt,enshu]{jarticle}  % ASCII
%\documentstyle[amstex,amssymb,showkeys,12pt,enshu]{jarticle}  % ASCII
%\fi

\documentclass[12pt,twoside]{jarticle}
\usepackage{amsmath,amssymb,amscd}
\usepackage{enshu}
\usepackage{mathrsfs}
\newcommand\scr{\mathscr}

\setcounter{page}{1}       % この数から始まる
\setcounter{section}{-1}   % この数の次から始まる
\setcounter{theorem}{0}    % この数の次から始まる
\setcounter{question}{0}   % この数の次から始まる
\setcounter{footnote}{0}   % この数の次から始まる

%%%%%%%%%%%%%%%%%%%%%%%%%%%%%%%%%%%%%%%%%%%%%%%%%%%%%%%%%%%%%%%%%%%%%%%%%%%
\begin{document}
%%%%%%%%%%%%%%%%%%%%%%%%%%%%%%%%%%%%%%%%%%%%%%%%%%%%%%%%%%%%%%%%%%%%%%%%%%%

\title{\bf 解析学演習}

\author{黒木 玄 \quad (東北大学理学部数学教室)}

\date{1995年11月6日(日)}

\maketitle

%%%%%%%%%%%%%%%%%%%%%%%%%%%%%%%%%%%%%%%%%%%%%%%%%%%%%%%%%%%%%%%%%%%%%%%%%%%
% 04-11.tex
%%%%%%%%%%%%%%%%%%%%%%%%%%%%%%%%%%%%%%%%%%%%%%%%%%%%%%%%%%%%%%%%%%%%%%%%%%%
% §. 演習の進め方
%%%%%%%%%%%%%%%%%%%%%%%%%%%%%%%%%%%%%%%%%%%%%%%%%%%%%%%%%%%%%%%%%%%%%%%%%%%

\section{演習の進め方}

この時間は, 微分積分学の演習を行なう. ただし, 講義の内容がそのまま演習
に反映されるとは限らないことを注意しておく. また, 以下のような思い込み
はすべて間違いである可能性が大きいので注意して欲しい:

\begin{enumerate}
\item 授業は教科書に沿って進められるのが当然であり, 教科書に書いてある
  ことはすべて正しいと思っている. 
\item 数学の演習とは, 練習問題を解きその答合わせをすることだと思ってい
  る. 答合わせをしないと心配で問題を解いた気にならない. 
\item 大学で習う数学は高校までに習った数学とスタイルが全然違うので, 自
  分は大学の数学に向いてないと思ってしまいやる気を無くしてしまった. 
\end{enumerate}

演習は以下のような方針て行なう:

\begin{itemize}
\item 問題が解けた人もしくは指名された人は黒板に解答を書きそれを発表す
  ること.  (発表の順番は私が黒板を見て決めます. ) そのとき, 問題の番号
  と自分の氏名・学籍番号を書くのを忘れないこと.
\item 数式だけの説明不足の解答は, 正式な解答とは認めない. 言葉を正確に
  用いて内容を詳しく説明した解答を書くこと. 
\item 問題が解けたと思って発表しても解答が完全でない場合は次の時間に再
  発表すること.  (すぐに修正できた場合はその時間中に再発表してもよい.)
\item すでに解かれてしまった問題でも, 別の方法で解けた場合はそれを発表
  してもよい. 
\item 演習問題を改良してから解いても良い. その改良が非常に良いものの場
  合は高く評価されるであろう.
\item なお, 演習問題自体が間違っている場合が多々あると思う. その場合は, 
  問題を適切に修正してから解くこと. 適切に訂正不可能な場合は, 反例など
  を挙げ, その理由を説明すること.
\end{itemize}

\bigskip

{\bf (参考書1)}\enspace 高木貞治著, 『解析概論』, 改訂第三版, 岩波書店.

この演習における問題の大部分は『解析概論』から切り出したものである.

{\bf (参考書2)}\enspace 寺沢寛一著, 『自然科学者のための数学概論』(増訂
版, 応用編の2冊がある), 岩波書店. この2冊は一般に『寺寛』という呼び名
で親しまれている.

数学科の学生は数学的な厳密な推論の仕方を入学時から叩き込まれることが多
い. そのため, 計算練習がおろそかになり, 大学の3年生頃には, 情けない状
態になっている人も出てくることが多い. そうならないためには, 機会あるご
とに面倒な計算にも挑戦しておくことが望ましい. しかし,意味のない計算問
題を計算練習のためだけのために解くのは精神的に大きな苦痛が伴なう.  精
神的な苦痛を減らすための一つの方法は, 計算結果に何らかの別の価値が伴な
うような計算をするよう心掛けることである. 例えば, 一般的な理論や定理に
おける基本的な例や反例について計算してみたり, 自然科学など現実を扱う分
野に現われる数学的なモデルに関する計算を実行してみたりするのである.こ
の後者のためには, 上記の寺沢寛一著の2冊が非常に役に立つ.

以上に挙げた, 高木貞治の『解析概論』や『寺寛』はいわゆる名著である. 時
がたっても数学の内容自身は本質的に変化しないので, 名著と呼ばれる数学の
本は時代を越えて役に立つ. 『解析概論』や『寺寛』は少なくとも孫子の代ま
では役に立つ. このような本に自分のお金を投資しても決して無駄にはならな
いと思う. どうせ買うなら投資価値の高い本を選ぶべきであろう.

{\bf (記号法)}\enspace 高校までの数学の教科書は文部省検定があるので記号
法は統一されているが, 今後はそのようなことはないので柔軟
に対処して欲しい. 例えば, $\leqq$ のことを $\le$ と書いたり %
$\leqslant$ と書いたりする. 特別な使いわけの規則などなく, 皆イーカゲン
に使っている.

%%%%%%%%%%%%%%%%%%%%%%%%%%%%%%%%%%%%%%%%%%%%%%%%%%%%%%%%%%%%%%%%%%%%%%%%%%%
% §. 基本的な概念
%%%%%%%%%%%%%%%%%%%%%%%%%%%%%%%%%%%%%%%%%%%%%%%%%%%%%%%%%%%%%%%%%%%%%%%%%%%

\section{数列の極限}

\begin{Definition}[数列の極限]
  数列 $a_n$ が $\alpha$ に収束するとは, 
  任意の $\varepsilon > 0$ に対して十分大きな $N$ を取って, $n \ge N$ 
  ならば $|a_n - \alpha| < \varepsilon$ が成立するようにできることであ
  る. このような $\alpha$ は, (存在するとすれば)数列 $a_n$ から一意的に
  定まり, 数列 $a_n$ の極限と呼ばれ, 
  $\displaystyle\lim_{n\to\infty}a_n=\alpha$ と表わされる.
\end{Definition}

\begin{question}
  上の数列の極限の定義の条件における $|a_n - \alpha| < \varepsilon$ を
  以下のように置き換えても, もとの条件と同値であることを示せ:
  \begin{enumerate}
  \item $|a_n - \alpha| \le \varepsilon$,
  \item $|a_n - \alpha| \le 3\varepsilon$.
  \qed
  \end{enumerate}
\end{question}

\noindent 数列の極限の定義において, $\varepsilon$ の前の不等号は $<$ 
および $\le$ のどちらでもよい. さらに, $\varepsilon$ の前に $3$ のよう
な任意の正の定数が挿入されてもよい. ちなみに, $<$ と $\le$ のどちらを
でもよい場合は, $\le$ の方を使った方が便利なことが多い.  なぜなら, 収
束する数列 $a_n$ が $a_n < A$ ($n=1,2,3,\ldots$) を満たしていても,
$\lim a_n < A$ が成立するとは限らないからである. ($a_n\le A$
($n=1,2,3,\ldots$) ならば $\lim a_n \le A$ が成立する.) 始めに $<$ を
使って出発しても, 極限操作をすることによって結局 $\le$ が出てくること
を避けられない場合が多いのである.

\begin{question}
  $0$ 以外の実数からなる数列 $a_n$ が $0$ でない実数に収束していると仮
  定する. このとき, 数列 $\displaystyle\frac{1}{a_n}$ も収束し,
  $\displaystyle
    \lim_{n\to\infty}\frac{1}{a_n}
    = \frac{1}{\lim\limits_{n\to\infty}a_n}
  $
  が成立することを示せ.
  \qed
\end{question}

\begin{question}
  $a>0$ ならば $\displaystyle\lim_{n\to\infty}\sqrt[n]{a}=1$.
  \qed
\end{question}

\begin{question}
  $a>1$, $k>0$ ならば $\displaystyle\lim_{n\to\infty}\frac{n^k}{a^n}=0$.
  \qed
\end{question}

\begin{question}
  $a>0$ ならば $\displaystyle\lim_{n\to\infty}\frac{a^n}{n!}=0$.
  \qed
\end{question}

\begin{question}\qstar{*}
  $\displaystyle\lim_{n\to\infty}a_n=\alpha$ ならば
  $\displaystyle\lim_{n\to\infty}\frac{a_1+a_2+\dots+a_n}{n}=\alpha$.
  \qed
\end{question}

\noindent 注意: 逆は成立しない. 例えば, $a_n = (-1)^{n-1}$ のとき,
$\displaystyle\lim_{n\to\infty}\frac{a_1+a_2+\dots+a_n}{n}=0$ だが, 
もとの $a_n$ 自身は収束しない. 

%%%%%%%%%%%%%%%%%%%%%%%%%%%%%%%%%%%%%%%%%%%%%%%%%%%%%%%%%%%%%%%%%%%%%%%%%%%

\section{実数の連続性}

\begin{Theorem}[実数の連続性]
  互いに同値な以下の条件のどれかによって, 実数の連続性が特徴付けられる:
  \begin{enumerate}
  \item 実数の切断は, 下組と上組との境界として, 一つの実数を確定する
    (Dedekindの定理)\footnote{『解析概論』定理1 (p.3)}.
  \item 数の集合 $S$ が上方[または下方]に有界ならば $S$ の上限[または
    下限]が存在する(Weierstrassの定理)\footnote{『解析概論』定理2 (p.5)}.
  \item 有界なる単調数列は収束する\footnote{『解析概論』定理6 (p.8)}.
  \item 閉区間 $I_n=[a_n,b_n]$ $(n=1,2,\ldots)$ において, 各区間 $I_n$ 
    がその前の区間 $I_{n-1}$ に含まれ,
    $\displaystyle\lim_{n\to\infty}|b_n-a_n|=0$ が成立するとき, これら
    の各区間には共通なる唯一の点が存在する(区間縮小法)\footnote{『解析
      概論』定理7 (p.10)}.
  \item 実数列 $a_n$ が収束するためには次の条件が成立すれば十分である
    (Cauchy列の収束, 実数全体の集合の距離空間としての完備性)\footnote
    {『解析概論』定理8 (p.11)}:
    \begin{description}
    \item[($\ast$)] 任意の $\varepsilon > 0$ に対して番号 $N$ をうまく
      定めると, $m \ge N$ かつ $n \ge N$ のとき$|a_n - a_m| <
      \varepsilon$ が成立する.
    \end{description}
    (この条件を満たす数列をCauchy列もしくは基本列と呼ぶ.)
  \end{enumerate}
\end{Theorem}

初学者はたくさんの同値な条件を挙げられて困惑をおぼえるであろう. 特に大
学に入ったばかりの学生の方々は, その論理的な複雑さについてゆけないもの
を感じ, 自信を失なってしまうかもしれない. しかし, そのような心配は無用
である. なぜなら, 実数の連続性の概念が上のような形で確定するまでには, 
多くの偉人達が膨大な時間をかけることが必要だったのである. その大変な努
力の結果だけを示され, すぐに理解しろと言われても, 困ってしまうのは当然
のことである. すぐに理解する必要はない. すぐに理解しようと無理をし, 理
解ができるまで先には絶対に進まないという精神で数学の勉強を続けることは
おそらく不可能であろうし, 可能であったとしても大変効率の悪いものになる
であろう. 「数学の本は後から読め!」という先人の言葉もあるように, 前か
ら順番に直線的に論理を追うだけの勉強法は止めた方が良い. 先に進みながら,
何度でも基本的なところに立ち戻って考えることが肝腎である. 

さて, 実数の連続性の話に戻ろう. 同値な条件をいくつか挙げたが, それらは
前半の3つ(Dedekindの定理, Weierstrassの定理, 単調な実数列の収束)と後半
の2つ(区間縮小法, Cauchy列の収束)に分類される. 前者の条件3つは主に不等
号 $<$ に関して実数全体の集合がどのような性質を持っているかに関係して
いる. 一方, 後者の2つは主に実数 $a$, $b$ の間の距離 $|b - a|$ に関して
実数全体の集合がどのような性質を持っているかに関係している. 前者は「全
順序集合」としての「完備性」の条件であり, 後者は「距離空間」としての
「完備性」を表現している. (ここで「」内に出た言葉の定義は未来の勉強に
よって学んで欲しい.)

\begin{question}
  Dedekindの定理からWeierstrassの定理を導け.  \qed
\end{question}

\begin{question}
  Weierstrassの定理から有界で単調な実数列の収束を導け. \qed
\end{question}

\begin{question}\qstar{*}
  有界で単調な実数列の収束から区間縮小法を導け. \qed
\end{question}

\begin{question}\qstar{*}
  区間縮小法からDedekindの定理を導け. \qed
\end{question}

\begin{question}\qstar{*}
  区間縮小法からCauchy列の収束を導け. \qed
\end{question}

\begin{question}\qstar{*}
  Cauchy列の収束から区間縮小法を導け. \qed
\end{question}

\noindent ヒント: 『解析概論』の第1章を見よ.

以下においては, 実数の連続性に関する結果は自由に用いて良い.

\begin{question}
  実数列 $a_n$ に対して, $\displaystyle\sum_{n=1}^\infty|a_n|$ が有限
  な値に収束しているならば, $\displaystyle\sum_{n=1}^\infty a_n$ も収
  束する. このとき, 級数 $\displaystyle\sum_{n=1}^\infty a_n$ は絶対収
  束すると言う.  \qed
\end{question}

\noindent ヒント: 『解析概論』の第42節(p.144). Cauchy列が収束すること
を使う.

%%%%%%%%%%%%%%%%%%%%%%%%%%%%%%%%%%%%%%%%%%%%%%%%%%%%%%%%%%%%%%%%%%%%%%%%%%%

\section{連続函数}

\begin{question}
  $f$, $g$ は $\R$ 上の実数値連続函数であるとする. このとき, $\R$ 上の
  函数 $h$ を $h(x)=g(f(x))$ によって定めると, $h$ も連続函数である.
  \qed
\end{question}

\begin{question}
  $\R$ 上の函数 $f$ を, $x\ne0$ のとき $f(x) = x \sin(1/x)$, 
  $f(0) = 0$ によって定める. このとき, $f$ は $\R$ 上の連続函数である. 
  ($\sin x$ が $\R$ 上の連続函数であることを使って良い.) 
  \qed
\end{question}

\begin{question}
  $\R^2$ 上の函数 $f$ を, 
  $(x,y)\ne(0,0)$ のとき $\displaystyle f(x,y)=\frac{2xy}{x^2+y^2}$, 
  $f(0,0) = 0$ によって定める. 
  このとき, $x$ もしくは $y$ の片方のみを固定し, $f(x,y)$ を $y$ もし
  くは $x$ の片方のみの函数であるとみなすとき, それは $\R$ 上の連続函数
  である. しかし, $k\ne0$ に対して, $t$ の函数 $f(t,kt)$ を考えると, 
  $t=0$ で連続ではない. 
  \qed
\end{question}

\noindent ヒント: 『解析概論』のp.26を見よ.

\begin{question}
  閉区間 $I = [0,1]$ を考える. $f$ は $I$ から $I$ への連続函数である
  と仮定する. このとき, 閉区間 $I$ 上の点 $x$ で $f(x)=x$ を満たすもの
  が存在する.
  \qed
\end{question}

\noindent ヒント: 中間値の定理を $F(x) = f(x) - x$ に適用する.

\noindent 参考: 一般に集合 $X$ から $X$ への写像 $f$ に対して,
$f(x)=x$ を満たす $x\in X$ を $f$ の不動点と呼ぶ. 上の問題のように, あ
る条件のもとで不動点の存在を保証する結果は不動点定理と呼ばれている. 上
の問題は高次元でも成立している Brouwer の不動点定理の最も簡単な場合
($1$ 次元の場合)である.

%%%%%%%%%%%%%%%%%%%%%%%%%%%%%%%%%%%%%%%%%%%%%%%%%%%%%%%%%%%%%%%%%%%%%%%%%%%
% 04-25.tex
%%%%%%%%%%%%%%%%%%%%%%%%%%%%%%%%%%%%%%%%%%%%%%%%%%%%%%%%%%%%%%%%%%%%%%%%%%%

%%%%%%%%%%%%%%%%%%%%%%%%%%%%%%%%%%%%%%%%%%%%%%%%%%%%%%%%%%%%%%%%%%%%%%%%%%%
% §. 基本的な概念 (続き)
%%%%%%%%%%%%%%%%%%%%%%%%%%%%%%%%%%%%%%%%%%%%%%%%%%%%%%%%%%%%%%%%%%%%%%%%%%%

\section{閉区間上の連続函数の基本性質}

この節の前半の目的は次の2つの定理を証明することである. (しかし, この節
の前半は, 厳密な解析学を学び始めたばかりの初学者にとっては結構難しいと
思われるので, 始めは結果だけを認めて先に進んでも良い.)

\begin{Theorem}\label{Th:max-min}
  閉区間上の実数値連続函数は最大値と最小値を持つ.
\end{Theorem}
%
\vspace{-\bigskipamount}
%
\begin{Theorem}\label{Th:unif-conti}
  閉区間上の連続函数は一様連続である.
\end{Theorem}

\noindent 微分学の基本定理であるRollの定理(平均値の定理の特別な場合)を
証明するために定理 \ref{Th:max-min}\ が使われる. 
また, 定理 \ref{Th:unif-conti}\ は, 閉区間上の連続函数がRiemannの意味
で積分可能であることを示すために役に立つ.

定理\ref{Th:max-min}を示すためには, 補題として次を示しておくと便利であ
る.

\begin{Theorem}[Bolzano-Weierstrass]\label{Th:BW}
  有界な実数列は収束する部分列を持つ.
\end{Theorem}

\begin{question}\qstar{*}
  定理 \ref{Th:BW}\ を証明せよ. \qed
\end{question}

\noindent ヒント: 『解析概論』の定理9 (p.14)の証明を1次元の場合に焼き直
せば良い.

\begin{question}\qstar{*}
  定理 \ref{Th:BW}\ が正しいことを認めた上で, 
  定理 \ref{Th:max-min}\ を証明せよ. 
  \qed
\end{question}

\noindent ヒント: 『解析概論』の定理13 (p.27)の証明を集積点という言葉を
使わない形に書き直せば良い. なお, この問題はBolzano-Weierstrassの定理 %
\ref{Th:BW}\ の証明を完成する前に解いて良いものとする.

定理 \ref{Th:unif-conti}\ を示すためには, 補題として次を示しておくと便
利である.

\begin{Theorem}[Heine-Borel]\label{Th:HB}
  閉区間 $K = [a,b]$ が開区間の族 $U_i = (a_i, b_i)$ ($i\in I$) によっ
  て覆われていると仮定する: $K \subseteq \bigcup\limits_{i\in I} U_i$.
  このとき, $K$ はすでに有限個の開区間で覆われている. 
  すなわち, 有限個の $i_1,\dots,i_n\in I$ を取って, 
  $K \subseteq U_{i_1} \cup \dots \cup U_{i_n}$ が成り立つようにできる.
\end{Theorem}

\begin{question}\qstar{*}
  定理 \ref{Th:HB}\ を証明せよ. \qed
\end{question}

\noindent ヒント: 『解析概論』の定理11 (p.16)の証明を今の場合に焼き直せ
ば良い.  (もちろん, 他にも色々な証明の仕方がある.)

\begin{question}\qstar{*}
  定理 \ref{Th:HB}\ が正しいことを認めた上で, 
  定理 \ref{Th:unif-conti}\ を証明せよ. 
  \qed
\end{question}

\noindent ヒント: 『解析概論』の定理14 (p.27)の証明よりも, 以下の証明の
方が標準的である. まず, $f$ は閉区間 $K$ 上の連続函数であると仮定し, %
$\varepsilon > 0$ を任意に固定する. 連続函数の定義より, %
任意の $x\in K$ に対して, ある $\delta(x) > 0$ を取って, %
任意の $y\in K$ に対して, %
$|y - x| < \delta(x)$ ならば $|f(y) - f(x)| < \varepsilon$ となるよう
にできる. %
$U_x = (x - \delta(x)/2, x + \delta(x)/2)$ と置くと, %
開区間の族 $\{U_x\}_{x\in K}$ は $K$ を覆っている. %
よって Heine-Borel の定理より, %
有限個の $x_1,\dots,x_n \in K$ を取って, %
$K \subseteq U_{x_1}\cup\dots\cup U_{x_n}$ が成り立つようにできる. %
$\delta = \max\{\delta(x_1),\dots,\delta(x_n)\}/2$ と置き, %
$|y - x| < \delta$ を満たす任意の $x,y\in K$ を固定する. %
このとき, $x$はある $U_{x_i}$ に含まれるので, %
$|x - x_i| < \delta(x_i)/2 < \delta(x_i)$ であり, %
$|y - x_i| \le |y-x|+|x-x_i| < \delta+\delta(x_i)/2 \le \delta(x_i)$ %
も成立する. %
よって, $|f(y)-f(x)| \le |f(y)-f(x_i)|+|f(x_i)-f(x)| < 2\varepsilon$. 
以上によって, $f$ の一様連続性が示された. (これは, ヒントと言いながら,
ほとんど完全な解答になってしまっている. この問題の解答者になった人は, 
図を描いたり, 言葉を付け加えて説明を詳しくするなどの工夫をして欲しい.)

\medskip
\noindent 注意: 以上においては, 閉集合や開集合という言葉を避けるために,
代わりに, 閉区間と開区間の場合に関する結果を述べた. 

\medskip
\begin{small}

以上の問題は, 厳密な解析学を習い始めたばかりの初学者にとっては, かなり
難解であると思われる. 『解析概論』における解説も結構繁雑でわかり難い. 
以上に挙げた定理をすっきり理解するためには, 位相空間および距離空間の一
般論を展開し, コンパクトという概念を抽象的に扱わなければいけない. 実際, 
閉区間はコンパクト空間の最も基本的な例であり, コンパクトという性質の定
義は, Heine-Borel の定理で示された閉区間の持つ性質を抽象化することによっ
てなされる. 例えば, 定理 \ref{Th:max-min}\ は, 以下の一般的な結果の簡単
な応用の一つになってしまうのである%
\footnote{しかし, 特別な場合に特別な証明を考えておくことは無駄なことで
  はないことを注意しておく. }:

\begin{itemize}
\item コンパクト空間の連続写像による像もまたコンパクトである.
\item 距離空間がコンパクトであるための必要十分条件は全有界かつ完備なこ
  とである.
\end{itemize}

\noindent この2つの結果から, ユークリッド空間の部分集合がコンパクトで
あることと有界閉集合であることが同値なことが導かれるのである. 最初に
「始めは結果だけを認めて先に進んでも良い」と述べたのは, このような事情
を考えてのことである. 

\end{small}

以上でこの節の前半を終え, 後半においては, 前半で述べた定理に関係する例
をいくつか挙げよう. 一つの定理を理解するためには, 定理の正しさを証明す
るだけでは不十分であり, それが成り立つ典型的な例にはどのようなものがあ
るか, 仮定の一部を落としたらどうなるかなど, 色々なことを考えておく必要
がある.

\begin{question}
  空でない開区間 $U=(a,b)$ の上には, 上にも下にも非有界な実数値連続函
  数が存在することを示せ.  \qed
\end{question}

%\noindent ヒント: 多くの例が考えられるが, 例えば, $\dfrac{1}{x-a}$ の
%ような函数を利用してみよ.

\begin{question}
  空でない開区間 $U=(a,b)$ の上には, (上にも下にも)有界だが, 最大値も
  最小値も持たない実数値連続函数が存在することを示せ. \qed
\end{question}

\begin{question}
  開区間 $U=(a,b)$ 上の連続函数 $f$ を考える. $x\in U$ が両端の %
  $a$, $b$ に近付くとき, $f(x)$ はある有限な値に収束していると仮定する. 
  このとき, $f$ は $U$ 上有界であることを示せ. \qed
\end{question}

\begin{question}
  空でない開区間 $U=(a,b)$ の上には, 連続だが一様連続でない連続函数が
  存在することを示せ.
\end{question}

\noindent ヒント: 非有界な連続函数を考えれば簡単に構成できる. しかし, 
できることなら, (上にも下にも)有界な連続函数でそのような例を作り, 非有
界な例と共に2つの例を示して欲しい. 有界な例を作るためには, 端の方で無
限に振動する連続函数の例を考えれば良い.

%%%%%%%%%%%%%%%%%%%%%%%%%%%%%%%%%%%%%%%%%%%%%%%%%%%%%%%%%%%%%%%%%%%%%%%%%%%
% §. 一変数函数の微分法
%%%%%%%%%%%%%%%%%%%%%%%%%%%%%%%%%%%%%%%%%%%%%%%%%%%%%%%%%%%%%%%%%%%%%%%%%%%

\section{微分と導函数}

\begin{question}
  $f$ は開区間 $U=(a,b)$ 上の函数であるとし, $x\in U$ を任意に固定
  する. 以下の条件は互いに同値であることを示せ:
  \begin{enumerate}
  \item 極限 $\displaystyle\lim_{h\to0}\frac{f(x+h)-f(x)}{h}$ %
    が存在する.
  \item ある $A$ が存在して, 任意の $\varepsilon > 0$ に対して,
    ある $\delta > 0$ が存在して, $x + h \in U$ かつ $|h|<\delta$ なら
    ば, $|f(x+h) - f(x) - A h| \le \varepsilon |h|$ が成立する.
    \label{enum:*/q:def-diff}
  \end{enumerate}
  このとき, $f$ の点 $x$ における微係数は $A$ に等しい.
  \qed
\end{question}

\noindent この同値な2つの条件の後者の方は次のように見ることができる. 
$f$ が $x$ で微分可能なとき, 
$R(x,h) = f(x + h) - f(x) - A h$ と置くと, 
\[
  f(x + h) = f(x) + Ah + R(x,h),
  \qquad
  \lim_{h\to 0}\frac{|R(x,h)|}{|h|} = 0
\]%
が成立する. %
これは $f(x + h)$ を $h$ の一次函数 $f(x) + Ah$ で近似する式であ
ると見ることができる. $\lim\limits_{h\to 0}\dfrac{|R(x,h)|}{|h|} = 0$ %
は誤差項 $R(x,h)$ が $h$ よりも高位の微小量であることを意味している. 
このような見方を $\varepsilon$-$\delta$ 論法で表現したのが後者の条件
(\ref{enum:*/q:def-diff})である. 条件 (\ref{enum:*/q:def-diff}) の方を
考えることには二つの利点がある. 一つは $h$ が $0$ であるかどうか気にす
る必要がないことである. このことは, 合成函数の微分法則を証明するとき, 
役に立つ. もう一つの利点は, $x$ が多変数になり $h$ がベクトルになった
場合でも, 条件(\ref{enum:*/q:def-diff}) は $A$ を $h$ に作用する線型写
像とみなすことによって一般化可能なことである.

%\begin{small}
%
%ついでに, $df$ や $dx$ などの記号がどのように合理化されるか説明しよう%
%\footnote{『解析概論』の第13節(pp.35--37)も参照せよ.}. %
%今度は $f$ は $U$ 上で微分可能であると仮定する. このとき, 上の記号の %
%$A$ は $f'(x)$ に等しい. $x$ に応じて変化する $h$ の一次函数 $df$ を %
%$df(x,h) = Ah = f'(x)h$ によって定義する. さらに, $h$ の函数 $dx$ を %
%$dx(h)=h$ によって定める. このとき,
%\[
%  df(x,h) = f'(x)h = f'(x)dx(h)
%\]%
%であるから, 両辺の $h$ と左辺の $x$ を略して次の式が得られる:
%\[
%  df = f'(x)dx.
%\]%
%これを $f$ の全微分と呼ぶ.
%
%\end{small}

\begin{question}[Leibniz 則]
  $(fg)'=f'g+fg'$ を用い, 自然数 $n$ に関する帰納法によって次を示せ:
  \[
    (fg)^{(n)} = \sum_{k=0}^n {n \choose k} f^{(k)} g^{(n-k)}.
  \]%
  ここで, $f^{(k)}$ は $f$ の $k$ 階の導函数を表わし, 
  ${n \choose k} = \frac{n(n-1)\cdots(n-k+1)}{k!}$ である. 
  \qed
\end{question}

\noindent ヒント: 二項係数の漸化式 %
${n \choose k} + {n \choose k+1} = {n+1 \choose k+1}$ を使う.

\begin{question}
  微分可能函数の導函数が連続とは限らないことを示せ%
\end{question}

\noindent ヒント: $f(x) = x^2 \sin (1/x)$.  

\noindent このような病的な函数を除外するために, 函数の微分可能性と共に
導函数の連続性も仮定することが多い. そこで, 次のような言葉を定義する.

\begin{Definition}
  微分可能函数は, その導函数が連続函数であるとき, $C^1$ 級函数と呼ばれ
  る.  より一般に, $n$ 回微分可能な函数は, その高階の導函数の全てが連
  続であるとき, $C^n$ 級函数と呼ばれる. 何回でも好きなだけ微分可能な函
  数は $C^\infty$ 級函数と呼ばれる.
\end{Definition}

\begin{question}\label{q:compact-supp-func}
  $a<b<c<d$ に対して, $\R$ 上の $C^\infty$ 級函数 $\psi$ で, %
  \[
    \text{
      $0 \le \psi \le 1$, \quad
      $b< x < c$ のとき $\psi(x) = 1$, \quad
      $x<a$ または $d<x$ のとき $\psi(x) = 0$
      }
  \]
  を満たすものが存在することを示せ. \qed
\end{question}

\noindent ヒント: $\R$ 上の函数 $f$ を $t>0$ のとき $f(t)=e^{-1/t^2}$, %
$t\le 0$ のとき $f(t)=0$ と定めると, $f$ は $C^\infty$ 級函数である. %
$g(t)=f(t)/(f(t)+f(1-t))$ と置くと, $g$ も $C^\infty$ 級で, %
$0 \le g \le 1$, $t < 0$ のとき $g(t)=0$, $t>1$ のとき $g(t)=1$ をみた
す. この $g$ を利用して $\psi$ を作ることを考えよ.

\begin{Definition}[線型常微分作用素]
  $a_0,a_1,\dots,a_n$ は開区間 $U$ 上の函数であるとする. $U$ 上の $n$ %
  回微分可能函数 $f$ を $U$ 上の函数 %
  $a_n f^{(n)} + a_{n-1} f^{(n-1)} + \cdots + a_1 f' + a_0 f$ %
  に対応させる写像を $U$ 上の高々 $n$ 階の線型常微分作用素と呼び, 
  \[
     P = a_n(x)\od{^n}{x^n} + a_{n-1}(x)\od{^{n-1}}{x^{n-1}} 
         + \cdots + a_1(x)\od{}{x} + a_0(x),
     \qquad
     Pf = \sum_{k=0}^n a_k f^{(k)}
  \] %
  のように書く. このように函数を函数に移す写像も我々は考えるのである.
\end{Definition}

\begin{question}
  上の定義の記号のもとで, 常微分作用素 $P$ が全ての多項式函数を %
  $0$ に移すならば, 全ての $a_0,a_1,\dots,a_n$ は恒等的に $0$ である.
  \qed
\end{question}

\noindent ヒント: 順次 $1, x, x^2, X^3, \ldots$ に $P$ を作用させて
みよ.

\noindent 上の定義の記号のもとで, 全ての $a_k$ が $C^\infty$ 級であれ
ば, $P$ は $C^\infty$ 級函数を $C^\infty$ 級函数に移す. このような線型
常微分作用素が二つ(例えば $P$, $Q$)あるとき, それらの写像の合成として, %
$C^\infty$ 級函数を $C^\infty$ 級函数に移す写像が得られ, その写像もま
た線型常微分作用素になる. この写像の合成によって,作用素の積が定義され
る: $(PQ)f := P(Qf)$. %
また, 作用素の和も自然に定義される: $(P + Q)f := Pf + Qf$.

\begin{question}[正準交換関係]
  函数 $f(x)$ を $f'(x)$ に移す作用素を $P = \od{}{x}$ と表わし, 函数 %
  $f(x)$ を $xf(x)$ に移す作用素を $Q = x$ と表わす. このとき, 作用素
  の積と和に関して, $PQ-QP=1$ が成立する. 特に, $PQ = QP$ は成立しない.
  \qed
\end{question}

\noindent ヒント: $(PQ-QP)f = P(Qf) - Q(Pf) = (xf)' - xf'$.

\begin{question}[Hermite多項式の定義]
  $n=0,1,2,\ldots$ に対して, $\R$ 上の函数 $H_n(x)$ を
  \[
    H_n(x) = e^{x^2} \left( - \od{}{x} \right)^n e^{-x^2}
  \] %
  によって定める. このとき, $H_n(x)$ は $n$ 次の多項式である. また, %
  $H_n(x)$ を次のように表わすこともできる:
  \[
    H_n(x) = e^{x^2/2} \left( - \od{}{x} + x \right)^n e^{-x^2/2}
  \]
  $H_n(x)$ は Hermite 多項式%
  \footnote{Hermite多項式は量子力学の調和振動子などの登場する有名な直
    交多項式である. }
  %
  と呼ばれている. \qed
\end{question}

\noindent ヒント: $n$ に関する帰納法. 後半を示すためには次の式を使えば
良い:
\[
  e^{x^2/2} \left( - \od{}{x} \right) ( e^{-x^2/2} f(x) )
  = \left( - \od{}{x} + x \right) f(x),
  \qquad
  H_{n+1}(x) = e^{x^2} \left( - \od{}{x} \right)(e^{-x^2}H_n(x)).
\]

\begin{question}\qstar{*}
  Hermite の多項式 $H_n(x)$ を用い, %
  函数 $f_n$ を $f_n(x)=H_n(x)e^{-x^2/2}$ と定め, %
  作用素 $H$ を %
  \( 
    H = -\frac{1}{2}\left(\od{}{x}\right)^2 + \frac{1}{2}x^2
  \) %
  と定める%
  \footnote{この $H$ が量子力学における調和振動子の Hamiltonian である.
    $p=-i\od{}{x}$ と置くと, %
    $H = \frac{1}{2}p^2 + \frac{1}{2} x^2$. %
    量子力学については, P.A.M.~Dirac著の邦訳『ディラック\ 量子力学\ 原
    書第4版』(岩波書店)を見よ. 調和振動子はその第VI章第34節で解説され
    ている.}.
  %
  このとき, %
  \( %
    Hf_n = \left( n + \frac{1}{2} \right)f_n
  \) %
  が成立する%
  \footnote{量子力学的には状態 $f_n$ のエネルギーが $n+\frac{1}{2}$ で
    あるということ.}. %
  \qed
\end{question}

\noindent ヒント: %
\( 
  A = \frac{1}{\sqrt{-2}} \left( \od{}{x} + x \right), \quad
  B = \frac{1}{\sqrt{-2}} \left( \od{}{x} - x \right), \quad
  N = B A, \quad
  \varphi_n = B^n f_0
\) %
と置くと, %
\break %
\( 
  H = N + \frac{1}{2}, \quad
  f_n = (-2)^{n/2} \varphi_n, \quad
  AB = BA + 1, \quad
  N \varphi_0 = B A \varphi_0 = 0
\). %
帰納法によって, %
\break %
\( %
  N B^n = B^n (N + n)
\) %
を示すことができる. %
\( %
  N \varphi_n = N B^n \varphi_0 = B^n(N + n) \varphi_0 
  = B^n n \varphi_0 = n \varphi_n
\).

%%%%%%%%%%%%%%%%%%%%%%%%%%%%%%%%%%%%%%%%%%%%%%%%%%%%%%%%%%%%%%%%%%%%%%%%%%%

\section{平均値の定理}

\begin{question}
  開区間上の函数 $f$ が $f(x) = A + B x$ ($A$, $B$ は定数)の形をしてい
  るための必要十分条件は, $f$ が微分可能でかつ $f$ の導函数が定数函数
  になることである.  \qed
\end{question}

\noindent ヒント: 平均値の定理を使えば良い. 

\noindent 注意: もしも, 微分積分法の基本公式%
\footnote{『解析概論』p.101の(1)式: $F'(x) = f(x)$ で $f$ が連続なとき %
  $\int_a^b f(x)\,dx = F(b) - F(a)$.}%
を使うことが許されるなら, 十分性の方を次のように簡単に証明することがで
きる. $f'(x) = B =(\text{定数})$ のとき,
\[
  f(x) = f(c) + \int_c^x f'(x)\,dx 
       = f(c) + \int_c^x B\,dx
       = f(c) + B (x - c)
       = A + B x.
\]%
ここで, $A = f(c) - B c$ と置いた. このように, 平均値の定理を使って証
明される結果は, 微分積分法の基本公式を使って証明できる場合が多い. 

\begin{question}
  $f$ は開区間 $U$ 上の微分可能函数であるとし, $U$ 上で $f' > 0$ であ
  ると仮定する. このとき, $f$ は $U$ 上狭義単調増大%
  \footnote{函数 $f$ が狭義単調増大するとは, $a < b$ ならば $f(a) <
    f(b)$ となるという意味である.}%
  する. \qed
\end{question}

\begin{question}
  $f$ は開区間 $U$ 上の $2$ 回微分可能函数であり, $U$ 上 $f'' \ge 0$ %
  であると仮定する. このとき, $f$ は $U$ において下に凸である. \qed
\end{question}

\noindent ヒント: 『解析概論』の定理25 (p.53)の(1).

\begin{question}[Rollの定理の拡張]
  $f$ は開区間 $(a,b)$ 上の微分可能函数であるとする. ただし, $a$ とし
  て $-\infty$, $b$ として $\infty$ も許す. 
  $x\to a$ および $x\to b$ のとき, $f(x)\to 0$ となると仮定する. この
  とき, ある $\xi\in(a,b)$ が存在して, $f'(\xi)=0$ が成立することを示
  せ. \qed
\end{question}

\noindent ヒント: 変数変換によって $a$, $b$ が共に有限な場合に帰着でき
る. $a$, $b$ が共に有限なとき, $f$ は $f(a)=f(b)=0$ と定義することによっ
て, 閉区間 $[a,b]$ 上の連続函数に拡張される.

\begin{question}\qstar{*}
  $f$ は開区間 $(a,b)$ 上の $C^\infty$ 級函数であるとする. ただし, $a$ 
  として $-\infty$, $b$ として $\infty$ も許す. %
  $n=0,1,2,\ldots$ に対して, $x\to a$ および $x\to b$ のとき %
  $f^{(n)} \to 0$ であり, 各 $f^{(n)}$ は高々 $n$ 個の零点%
  \footnote{函数 $f$ の零点とは $f(x)=0$ を満たす $x$ のことである.}%
  しか持たないと仮定する. このとき, 以下が成立する:
  \begin{enumerate}
  \item $f^{(n)}$ はちょうど $n$ 個の相異なる零点を持つ.
  \item $f^{(n)}$ の $n$ 個の相異なる零点を %
    $\xi_{n,1} < \xi_{n,2} < \dots < \xi_{n,n}$ と表わすと,
    $f^{(n-1)}$ と $f^{(n)}$ の零点の間には %
    $\xi_{n,k} < \xi_{n-1,k} < \xi_{n,k+1}$ という関係がある.  \qed
  \end{enumerate}
\end{question}

\noindent ヒント: Rollの定理およびその拡張を使うことを考えよ. $n$ に関
する帰納法を使う.

\begin{question}[Hermite多項式の零点の分布]
  非負の整数 $n$ に対して, $x$ の函数 $H_n(x)$ を次の式によって定義す
  る:
  \[
    \left( - \od{}{x} \right)^n e^{-x^2} = H_n(x) e^{-x^2}
  \]%
  このとき, $H_n(x)$ は $n$ 次の多項式となり, $n$ 個の相異なる実根を持
  ち, それらは $H_{n-1}(x)$ の実根によって隔離される.
  \qed
\end{question}

\noindent ヒント: すぐ上の問題の結果を用いて良い. (『解析概論』の第2章
の練習問題の(1),(2),(3) (p.84)も参照せよ.)

\begin{question}[Newtonの近似法]\qstar{*}
  $f$ は閉区間 $[a,b]$ 上の2回微分可能な函数であり, %
  $f(a) > 0$, $f(b) < 0$ であり, $a\le x \le b$ において $f''(x) > 0$ %
  を満たしていると仮定する. このとき $[a,b]$ 内の数列 $a_n$ を帰納的に
  次のように定めることができる.  まず $a_1 = a$ と置き, $a_n$ が定まっ
  たとき $f$ の $x=a_n$ における接線の零点を $a_{n+1}$ と置く: %
  \(\displaystyle
    a_{n+1} = a_n - \frac{f(a_n)}{f'(a_n)}.
  \) %
  このとき, 数列 $a_n$ は $[a,b]$ における方程式 $f(x) = 0$ の唯一の解
  に単調に収束する.  \qed
\end{question}

\noindent ヒント: 『解析概論』の第2章の練習問題(8) (p.85). グラフを描
いてみると納得し易い.

\noindent 注意: $f(a) < 0$, $f(b) > 0$ の場合は, $a_1$ として $a$ の代
わりに $b$ を使えば良い. $f''(x) < 0$ の場合は $f$ の代わりに $-f$ を
考えれば良い.

\begin{question}
  Newtonの近似法を認め, $\sqrt{2}$ の近似値を次のようにして数値計算せ
  よ. まず, $f(x) = x^2 - 2$, $a=0$, $b=2$ と置く. $a_1 = b$ から出発
  し, $a_2,a_3,a_4$ を計算し, $a_4$ が $\sqrt{2}$ と小数点以下5桁まで
  一致していることを示せ.  \qed
\end{question}

%%%%%%%%%%%%%%%%%%%%%%%%%%%%%%%%%%%%%%%%%%%%%%%%%%%%%%%%%%%%%%%%%%%%%%%%%%%

\section{Taylorの定理}

この節では平均値の定理の拡張である Taylor の定理(公式)を扱う. 以下にお
いて, $n$ は自然数であるとし, $f$ は開区間 $U$ 上の $n$ 回微分可能な函
数であるとする. $x, x + h \in U$ と $k = 1,2,\dots,n+1$ に対して, %
$x$, $h$ の函数 $R_k = R_k(x,h)$ を以下の式によって定義する:
\[
    f(x + h)
    = f(x)
    + \frac{1}{1!}     f'(x)        h
    + \frac{1}{2!}     f''(x)       h^2
    + \cdots
    + \frac{1}{(k-1)!} f^{(k-1)}(x) h^{k-1}
    + R_k(x,h).
\]
Tayorの定理とは剰余項 $R_k(x,h)$ に関する結果である.

\begin{Theorem}[Taylorの定理]\label{Th:Taylor1}
  以上の記号のもとで, $x$, $h$ に対して, %
  $0 < \theta < 1$ なるある実数 $\theta$ が存在して, %
  $R_n(x,h)$ は次のように表わされる:
  \[
    R_n(x,h) = \frac{1}{n!} f^{(n)}(x + \theta h) h^n.
  \]%
  さらに, $R_{n+1}$ は次を満たす:
  \[
    \lim_{h\to 0} \frac{|R_{n+1}(x,h)|}{|h|^n} = 0
    \qquad (\text{すなわち}\quad R_{n+1} = o(|h|^n)).
  \]
\end{Theorem}

\begin{question}\qstar{*}
  定理 \ref{Th:Taylor1}\ を証明せよ. \qed
\end{question}

\noindent ヒント: 『解析概論』の第25節 (p.61).

平均値の定理を使うと $f$ の導函数の値の正負によって, $f$ の増減の様子
が判定できるのであった. その拡張である Taylor の定理を使うと高階の微係
数から, 函数の局所的な様子に関する情報を引き出すことができる. 例えば, 
次が成立する.

\begin{question}
  $n$ は正の偶数であるとし, $f$ は開区間 $U$ 上の $n$ 回微分可能函
  数であるとする. ある $a\in U$ において, %
  $f'(a) = f''(a) = \cdots = f^{(n-1)}(a) = 0$, $f^{(n)}(a) \ne 0$ で
  あると仮定する. %
  このとき, $f^{(n)}(a) > 0$ ならば $f$ は $a$ で極小値を取り, %
  $f^{(n)}(a) < 0$ ならば $f$ は $a$ で極大値を取る. \qed
\end{question}

\noindent ヒント: Taylor の定理の $R_{n+1}$ に関する結果を使う. (極大・
極小の定義については『解析概論』の第26節 (p.67)を参照せよ.)

毛色の変わった応用として次のような問題も考えられる.

\begin{question}\qstar{*}
  $e$ が無理数であることを証明せよ. \qed
\end{question}

\noindent ヒント: 『解析概論』のp.66. 

%%%%%%%%%%%%%%%%%%%%%%%%%%%%%%%%%%%%%%%%%%%%%%%%%%%%%%%%%%%%%%%%%%%%%%%%%%%
% §. 一変数函数の積分法
%%%%%%%%%%%%%%%%%%%%%%%%%%%%%%%%%%%%%%%%%%%%%%%%%%%%%%%%%%%%%%%%%%%%%%%%%%%

\section{積分の基本的性質}

積分を定義する方法には色々な流儀がある. 完壁を期すならば, Lebesgue 積
分論を展開しなければいけない. しかし, Lebesgue 積分論の展開のためには, 
測度論を前もって展開しておく必要があり, 一変数の初等微分積分学を展開す
る上では大袈裟に過ぎるような感じがする. 一方, Riemann 積分の方はある程
度初等的に理論を展開できるが, Lebesgue 積分論によって得られるようなすっ
きりした結果を得ることはできない. 他にも, S.~Lang の本
``Real Analysis''%
\footnote{その日本語訳が共立出版から出ている. 書名は『現代の解析学』.}%
の第5章の第1節のように, 閉区間上の階段函数の一様収束極限で書けるような
函数のクラスに制限し, 簡明に積分を定義してしまうという流儀もある.

このように, 積分論の展開の流儀は色々あるのだが, 通常現われるような「良
い函数」に対しては, 積分をどの流儀で計算してもその値はもちろん一致して
いる.  だから, 積分が現われる問題を解くときには, 流儀によらない積分の
基本的性質のみを使うように努力した方が良い. 積分の基本的性質とは例えば
以下のような結果のことである:
%
\begin{itemize}
\item $\int_a^b 1\,dx = b - a$.
\item 線型性: 定数 $\alpha$, $\beta$ に対して, %
  \(
    \int_a^b (\alpha f(x) + \beta g(x)) \,dx
    = \alpha \int_a^b f(x) \,dx + \beta \int_a^b g(x) \,dx
  \).
\item 区間に関する加法性: %
  $\int_a^c f(x) \,dx = \int_a^b f(x) \,dx + \int_b^c f(x) \,dx$.
\item 単調性: $a \le b$ かつ $f \le g$ ならば, %
  $\int_a^b f(x) \,dx \le \int_a^b g(x) \,dx$.
\item $a \le b$ ならば $|\int_a^b f(x) \,dx| \le \int_a^b |f(x)| \,dx$.
\item 連続函数は閉区間上で積分可能である.
\end{itemize}
%
もちろん, これらの結果は積分の定義の流儀によらずに成立している. 

\begin{question}
  上記の積分の基本的性質のみを用いて次を示せ. $f$ は開区間 $U$ 上の連
  続函数であるとし, %
  $x,c\in U$ に対して, $F(x) = \int_c^x f(\xi)\,d\xi$ と置く. %
  このとき, $F$ は微分可能であり, $F' = f$ が成立する. %
  また, $a,b\in U$ に対して, $\int_a^b f(x) \,dx = F(b) - F(a)$ も成立
  している. \qed
\end{question}

\noindent ヒント: 『解析概論』の第32節(p.101).

\begin{question}
  上記の積分の基本性質のみを用いて次を示せ. $f$ が閉区間 $[a,b]$ 上の
  連続函数であるとき,
  \[
    \left| \int_a^b f(x) \,dx \right| 
    \le
    |b - a|\,\sup_{a\le x\le b}|f(x)|
    < \infty.
    \qed
  \]
\end{question}

ある区間上の複素数値函数 $h(x) = f(x) + i g(x)$ ($f$, $g$ は実数値函数)
の微分と積分を
\[
  h'(x) = f'(x) + i g'(x),
  \qquad
  \int_a^b h(x)\,dx = \int_a^b f(x)\,dx + i \int_a^b g(x)\,dx
\]%
と定義しておく. 実際には複素数値函数(より一般にはベクトル値函数)の微分
と積分を実数値函数を経由せずに直接定義した方が奇麗なのだが, ここでは, 
ひとまずこのように定義しておく.

\begin{question}
  この問題を解くために, 複素数 $z = x + yi$ ($a,b\in\R$) に対して, %
  $e^z = e^x(\cos y + i \sin y)$ が成立するという事実を用いて良い. %
  複素数 $c$ に対して, $\R$ 上の複素数値函数 $f$ を $f(x)=e^{cx}$ によっ
  て定めると, $f'(x)=ce^{cx}$ が成り立つ. よって, $c\ne 0$ %
  のとき, 
  \[
    \int e^{cx} \,dx = \frac{e^{cx}}{c}.
  \]%
  この公式の実部と虚部を分離して書くと, 次が成立することもわかる. %
  $a,b\in\R$ に対して, $(a,b)\ne(0,0)$ ならば,
  \[
    \int e^{ax} \cos(bx) \,dx
    =
    e^{ax}
    \frac{a \cos(bx) + b \sin(bx)}{a^2+b^2},
  \quad
    \int e^{ax} \sin(bx) \,dx
    =
    e^{ax}
    \frac{a \cos(bx) - b \sin(bx)}{a^2+b^2}.
  \qed
  \]
\end{question}

\noindent 参考: 『解析概論』の第35節の例3 (p.115)に, 部分積分を用いた別
の導出の仕方が書いてある. しかし, その計算はかなり技巧的でわかり難い.

%%%%%%%%%%%%%%%%%%%%%%%%%%%%%%%%%%%%%%%%%%%%%%%%%%%%%%%%%%%%%%%%%%%%%%%%%%%

\section{広義積分}

前節では閉区間上の連続な(したがって有界な)函数の積分を扱った. この節で
は区間の境界の近くで有界でない函数および無限に長い区間上での積分を扱う.

\begin{Definition}
  $f$ は区間 $[a,b)$ ($b$ として $\infty$ も許す)上の連続函数であり, 
  極限 $\lim\limits_{\beta\nearrow b}\int_a^\beta f(x)\,dx$ が存在する
  とき, その値を $f$ の広義積分(improper integral)と呼び, %
  $\int_a^b f(x)\,dx$ と略記する.  区間 $[a,b)$ の代わりに $(a,b]$ を
  取る場合も同様に広義積分が定義される. 区間 $(a,b)$ 上の広義積分は %
  $c\in(a,b)$ を任意にとって, $(a,c]$ と $[c,b)$ の上の広義積分に分け
  て考え, それらの和を取ることによって定義される.
\end{Definition}

\begin{Example}
  広義積分の簡単な例:
  \begin{enumerate}
  \item \( \displaystyle
    \int_0^1 \frac{dx}{\sqrt{1-x^2}}
    = \lim_{a\nearrow1} \int_0^a \frac{dx}{\sqrt{1-x^2}}
    = \lim_{a\nearrow1}\, \left[ \arcsin x \right]_0^a
    = \lim_{a\nearrow1} \arcsin a
    = \frac{\pi}{2}
    \).
  \item \( \displaystyle
    \int_0^\infty \frac{dx}{x^2}
    = \lim_{R\to\infty} \int_0^R \frac{dx}{x^2}
    = \lim_{R\to\infty} \left[ -\frac{1}{x} \right]_1^R
    = \lim_{R\to\infty} \left( 1 - \frac{1}{R} \right)
    = 1
    \).
  \end{enumerate}
\end{Example}

\begin{question}[Cauchyの収束判定法の連続変数版]
  $f$ が区間 $(a,b)$ 上の実数値函数であるとき, 極限 %
  $\lim\limits_{x\nearrow b} f(x)$ が存在するための必要十分条件は, 次が成
  立することである:
  \begin{description}
  \item[($\ast$)] 任意の $\varepsilon > 0$ に対して, %
    ある $c\in(a,b)$ が存在して, $x,y \ge c$ なる任意の $x,y \in (a,b)$%
    に対して, $|f(x) - f(y)| \le \varepsilon$ が成立する.
    \qed
  \end{description}
\end{question}

\noindent ヒント: 必要性の方は簡単. 十分性の方は背理法を使って示せる. 
つまり, ($\ast$) および極限 $\lim\limits_{x\nearrow b} f(x)$ が存在し
ないことを仮定すると, $b$ に単調に収束する $(a,b)$ 内の点列 $x_n$ で, %
$f(x_n)$ がCauchy 列になるにもかかわらず収束しないものが取れることを示
す.

\begin{question}[広義積分の絶対収束の定義]
  $f$ は区間 $[a,b)$ ($b$ として $\infty$ も許す)上の連続函数であると
  する. $|f|$ の $[a,b)$ における広義積分が存在すれば, $f$ 自身の広義
  積分も存在することを示せ. このとき, $f$ の $[a,b)$ における広義積分
  は絶対収束すると言う. また, 絶対収束しない広義積分は条件収束すると言
  う. \qed
\end{question}

\noindent ヒント: 数列の収束の Cauchy の判定法の連続変数版を使えば良い.

\begin{small}

\noindent 参考: Lebesgue 積分論を展開すれば, 被積分函数もしくは積分範
囲が有界でない積分を広義積分として特別扱いする必要はなくなる. さらに, 
絶対収束する広義積分の値と Lebesgue 式の積分による値は等しくなる. 
したがって, 絶対収束する広義積分の理論は Lebesgue 積分論に含まれてしま
うと考えて良い. しかし, 絶対収束しない広義積分は Lebesgue 積分論を展開
した後でも特別扱いする必要がある.

\end{small}

\begin{question}
  広義積分が絶対収束するための十分条件として以下のような判定法がある:
  \begin{enumerate}
  \item $f$ が有限区間 $[a,b)$ 上の連続函数であるとする. %
    $0 < \alpha < 1$ なるある $\alpha$ に関して %
    $|x - b|^\alpha |f(x)|$ が有界ならば, %
    広義積分 $\int_a^b f(x)\,dx$ は絶対収束する.
  \item $f$ が半無限区間 $[a,\infty)$ 上の連続函数であるとする. %
    ある $\alpha > 1$ に関して $|x|^\alpha |f(x)|$ が有界ならば, %
    広義積分 $\int_a^\infty f(x)\,dx$ は絶対収束する.
    \qed
  \end{enumerate}
\end{question}

\noindent ヒント: 『解析概論』の定理36 (p.106).  (この結果は以下では自
由に用いて良い.)

\begin{question}[ガンマ函数]
  $s > 0$ のとき, 次の広義積分は絶対収束する:
  \[
    \Gamma(s) = \int_0^\infty e^{-x} x^{s-1} dx.
  \]
  この式によって定義される函数 $\Gamma(s)$ はEulerのガンマ函数と呼ばれ
  る. \qed
\end{question}

\begin{question}[ベータ函数]
  $p > 0$, $q > 0$ のとき, 次の広義積分は絶対収束する:
  \[
    B(p,q) = \int_0^1 x^{p-1} (1 - x)^{q-1} \,dx.
  \]
  この式によって定義される函数 $B(p,q)$ はEulerのベータ函数と呼ばれる.
  \qed
\end{question}

\noindent 参考: ベータ函数はガンマ函数によって, %
$B(p,q) = \dfrac{\Gamma(p)\Gamma(q)}{\Gamma(p+q)}$ と表わされる. しか
し, これを証明するためには二重積分の積分変数の変換公式が必要になる. 

\begin{question}
  広義積分 $\displaystyle \int_0^\infty\frac{\sin x}{x}\,dx$ は収束する
  が絶対収束しない. \qed
\end{question}

\noindent ヒント: 『解析概論』のp.106の例. 

\begin{question}[超幾何積分]\label{q:hypergeom-int}
  $\alpha > 0$, $\gamma-\alpha > 0$, %
  $\gamma \ne 0,-1,-2, \dots$, $x<1$ のとき%
  \footnote{実際には, $\alpha$, $\beta$, $\gamma, x$ は複素数でも良い.
    ただし, その場合は, 広義積分を絶対収束させるための条件は, %
    $\Repart\alpha > 0$, $\Repart(\gamma-\alpha) > 0$, %
    $x\notin[1,\infty)$ になる.}, % 
  次の式の右辺の広義積分は絶対収束する:
  \[
    F(\alpha,\beta,\gamma; x)
    =
    \frac{\Gamma(\gamma)}{\Gamma(\alpha)\Gamma(\gamma-\alpha)}
    \int_0^1 t^{\alpha-1} (1 - t)^{\gamma-\alpha-1} (1 - xt)^\beta \,dt.
  \]%
  これを, 超幾何積分もしくは超幾何函数%
  \footnote{超幾何函数(hypergeometric function)については, 青本和彦・
    喜多通武共著の『超幾何関数論』(シュプリンガー・フェアラーク東京)を
    参照せよ. 超幾何函数の積分表示については, その本の p.11 に解説があ
    る. 超幾何函数の性質がよくわかるのは Euler 型の積分表示式を持つか
    らであり, これは超幾何函数の最も重要な性質である.}%
  の Euler 型積分表示式と呼ぶ.
  \qed
\end{question}

\noindent 参考: 実はこの積分の値は $|x|<1$ において問題 %
\qref{q:hypergeom-series}\ で定義される超幾何級数に等しい%

%%%%%%%%%%%%%%%%%%%%%%%%%%%%%%%%%%%%%%%%%%%%%%%%%%%%%%%%%%%%%%%%%%%%%%%%%%%

\section{部分積分}

数学解析の三種の神器は, Taylor展開, 部分積分, Fourier変換であるとされ
ている(らしい(?)). この節では, 二番目の部分積分を扱うことにしよう.

\begin{question}[integration by parts]
  $f$, $g$ は開区間 $U$ 上の微分可能函数であり, それらの導函数 %
  $f'$, $g'$ は連続であると仮定する. このとき, $a,b\in U$ に対して以下
  の公式が成立することを証明せよ:
  \[
    \int_a^b f(x)g'(x)\,dx
    = \left[ f(x)g(x) \right]_a^b - \int_a^b f'(x)g(x)\,dx.
  \]
  特に, $f(a)g(a) = f(b)g(b)$ のとき,
  \[
    \int_a^b f(x)g'(x)\,dx = - \int_a^b f'(x)g(x)\,dx.
    \qed
  \]
\end{question}

\noindent 参考: 部分積分はこの問題の最後の形で使われることが多い. 後で
触れる超函数の意味での導函数はこの形での部分積分の公式を先取りすること
によって定義される.

\begin{question}[ガンマ函数の函数等式]
  部分積分を用いて, ガンマ函数の函数等式 $\Gamma(s+1)=s\Gamma(s)$ を証
  明せよ. この結果と $\Gamma(1)=1$ を合わせると, $n=0,1,2,\ldots$ に対
  して, $\Gamma(n+1)=n!$ が成立することがわかる. \qed
\end{question}

\noindent ヒント: 『解析概論』の第35節の例4 (p.115).

\noindent 参考: $\Gamma(\frac{1}{2})=\sqrt{\pi}$ が成立している%
\footnote{公式 $\int_0^\infty e^{-x^2}\,dx=\sqrt{\pi}/2$ (問題 %
  \qref{q:int-exp-x2})において, $t = x^2$ と変数変換すればこの
  式が得られる.}. %
この結果を認めると, 函数等式によって, $n=0,1,2,\ldots$ に対して, %
\(
  \Gamma\left( n + \frac{1}{2} \right)
  = \frac{1 \cdot 3 \cdot 5 \cdots (2n - 1)}{2^n} \sqrt{\pi}
\) %
が成立することがわかる.

\begin{question}[Wallisの公式1]
  $n=0,1,2,3,\ldots$ に対して, %
  $S_n = \int_0^{\pi/2} \sin^n \theta \,d\theta$ と置くと, 部分積分に
  よって, $S_n = \dfrac{n-1}{n} S_{n-2}$ ($n \ge 2$) を示すことができ
  る. 簡単な計算で, $S_0 = \pi/2$, $S_1 = 1$ となることがわかるので, 
  帰納法によって次が成立することが示される:
  \begin{align*}
    S_{2n}
    &
    =
    \frac{1}{2} \cdot \frac{3}{4} \cdot \frac{5}{6} \cdots \frac{2n-1}{2n}
    \cdot \frac{\pi}{2},
    \\
    S_{2n+1}
    &
    =
    \frac{2}{3} \cdot \frac{4}{5} \cdot \frac{6}{7} \cdots \frac{2n}{2n+1}.
  \end{align*}
  これを Wallis の公式と呼ぶ. \qed
\end{question}

\begin{question}[Wallisの公式2]\qstar{*}
  上の問題の記号のもとで, %
  $\displaystyle \lim_{n\to\infty} \frac{S_{n+1}}{S_n} = 1$ %
  となることを証明せよ. この結果を変形して, 以下の公式を導け:
  \begin{enumerate}
  \item \( \displaystyle
    \prod_{n=1}^\infty \left(1 - \frac{1}{(2n)^2} \right)
    = \frac{2}{\pi}
    \).
  \item \( \displaystyle
    \lim_{n\to\infty} \frac{2^{2n}(n!)^2}{\sqrt{n}(2n)!} = \sqrt{\pi}
    \).
  \end{enumerate}
  これらの公式もWallisの公式と呼ばれる. \qed
\end{question}

\noindent ヒント: 以上の2問は『解析概論』の第35節の例5 (p.116)の内容で
ある.

%%%%%%%%%%%%%%%%%%%%%%%%%%%%%%%%%%%%%%%%%%%%%%%%%%%%%%%%%%%%%%%%%%%%%
%もしも rsfs フォントを持ってなければ次の行のコメントアウトをはずせ.%
%%%%%%%%%%%%%%%%%%%%%%%%%%%%%%%%%%%%%%%%%%%%%%%%%%%%%%%%%%%%%%%%%%%%%
%\def\scr{\cal}

\begin{Definition}[超函数の意味での導函数]\label{Def:weak-der}
  $U$ は任意の開区間とし, $\scr{D}(U)$ によって $U$ 上の $C^\infty$ 級
  函数で $U$ 内のある閉区間の外で恒等的に $0$ になるもの全体の集合
  を表わすことにする%
  \footnote{すなわち, $\scr{D}(U)$ の元はコンパクトな台を持つ $U$ 上の 
    $C^\infty$ 級函数である. 例えば, 問題 \qref{q:compact-supp-func}\ %
    における函数 $\psi$ は $\scr{D}(\R)$ の元である. また, その $\psi$ 
    の開区間 $U=(a,d)$ への制限は $\scr{D}(U)$ の元である. }. %
  $f$, $T$ が $U$ 上の函数であるとき, $T$ が $f$ の超函数の意味での
  導函数であるとは, %
  任意の $\varphi \in \scr{D}(U)$ に対して, 
  \[
    \int_U T(x) \varphi(x) \,dx = - \int_U f(x) \varphi'(x) \,dx.
  \]%
  が成立することであると定義する. ただし, ここで, 積分は $\varphi$ が
  零にならない点の全体を含む任意の閉区間上の定積分を意味するものとする.
  (そのような閉区間の取り方に積分の値はよらない.)
\end{Definition}
%
このように, 超函数の意味での導函数は部分積分の公式の一般化として定義さ
れる.

\begin{question}
  $\R$ 上の函数 $H(x)$ を, $x < 0$ のとき $H(x) = 0$, $x > 0$ のとき %
  $H(x) = 1$, $H(0) = \text{(任意の数)}$ と定義する. さらに, $f(x)$ を, %
  $x < 0$ のとき $f(x) = 0$, $x \ge 0$ のとき $f(x) = x$ と定義する. 
  このとき, $H(x)$ は $f(x)$ の超函数の意味での導函数である.  
  一般に, $H(x)$ は Heaviside 函数と呼ばれている.
  \qed
\end{question}

\noindent 参考: さらに Heaviside 函数を超函数の意味で微分するとどうな
るのであろうか? もはや, Heaviside 函数の超函数の意味での導函数は普通の
函数では表現できず, Schwartz の超函数(distribution)を導入しなければいけな
い. なお, Heaviside 函数の導超函数は, Dirac のデルタ(超)函数というもの
になる.  Dirac のデルタ函数 $\delta(x)$ の基本性質は,
\[
  \int_{-\infty}^\infty \delta(x)\varphi(x) \,dx = \varphi(0)
\]%
である. (Heaviside 函数の導函数が形式的にこのデルタ函数の性質を満たし
ていることを示してみよ.)  実は, Schwartz 流の超函数論ではこの式をもっ
てデルタ函数の定義とするのである. なお, Schwartz の流儀とは全く異なる
方法で超函数を定義する佐藤の超函数(hyperfunction)の理論もあることを注
意しておく.

\begin{question}[変分法の基本補題]
  任意の開区間 $U$ に対して, $U$ 上の $C^\infty$ 級
  函数で $U$ 内のある閉区間の外で恒等的に $0$ になるもの全体の集合
  を $C_0^\infty(U)$ と表わすことにする%
  \footnote{もちろん, $\scr{D}(U) = C_0^\infty(U)$ である. 記号 %
    $\scr{D}$ の方は主に Schwartz の超函数の試験函数の空間を意味する場
    合に使われる. コンパクト台を持つ $C^\infty$ 級函数の空間を表わす記
    号として, $C_0^\infty$ の方がより広く使われている.}. %
  $f$ は開区間 $U=(a,b)$ 上の連続函数であるとする. %
  このとき, 全ての $\varphi\in C_0^\infty(U)$ に対して%
  $\int_a^b f(x) \varphi(x) \,dx = 0$ が成立するならば, %
  $f$ は $U$ 上恒等的に $0$ である. \qed
\end{question}

\noindent ヒント: $f(x)\ne 0$ となる $x\in U$ が存在するとき, $x$ のあ
る近傍の外で $0$ となるような $\varphi\in C_0^\infty(U)$ をうまくとって, 
$\int_a^b f(x)\varphi(x)\,dx\ne 0$ が成り立つようにできることを示す. 
問題 \qref{q:compact-supp-func}\ の結果を利用せよ. 

%\noindent ヒント: $f$ が $0$ にならない点の近傍の外で $0$ になるような
%問題 \qref{q:compact-supp-func}\ における $\psi\in C_0^\infty(U)$ を
%うまく取って, $\int_a^b f(x)\psi(x) \,dx \ne 0$ となるようにできるこ
%とを示す.

%%%%%%%%%%%%%%%%%%%%%%%%%%%%%%%%%%%%%%%%%%%%%%%%%%%%%%%%%%%%%%%%%%%%%%%%%%%

\section{積分変数の変換}

\begin{question}[不定積分の計算]
  以下 $a\ne0$, $n=1,2,3,\ldots$ とする.
  \begin{enumerate}
  \item $\displaystyle I_n := \int\frac{dx}{(x^2+a^2)^n}$ と置く.
    このとき, $I_1 = \dfrac{1}{a} \arctan \dfrac{x}{a}$ および次が成立
    する:
    \[
      I_{n+1}
      = \frac{1}{2na^2}\frac{x}{(x^2+a^2)^n} + \frac{2n-1}{2na^2} I_n.
    \] %
    (ヒント: 被積分函数に $1 = x'$ が隠れているとみなし部分積分する.)
  \item 
    \( \displaystyle
        \int \frac{dx}{\sqrt{x^2+a^2}}
      = \log(x + \sqrt{x^2+a^2}).
    \) %
    \quad (ヒント: $t - x = \sqrt{x^2+a^2}$.)
  \item 
    \( \displaystyle
      \int \frac{dx}{\sqrt{x^2-a^2}}
      = \log \left| x + \sqrt{x^2-a^2} \right|.
    \) %
    \quad (ヒント: $t = \sqrt{\dfrac{x-a}{x+a}}$.)
  \item 
    \( \displaystyle
      \int \sqrt{a^2-x^2}\,dx
      = \frac{1}{2}
        \left( x\sqrt{a^2-x^2} + a^2\arcsin\frac{x}{a} \right).
    \) %
    \quad (ヒント: $I := \int \sqrt{a^2-x^2}\,dx$ と置く. $1=x'$ が
    隠れているとみなし部分積分することによって, 
    \[
      I = x \sqrt{a^2-x^2} - I + a^2 \int \frac{dx}{\sqrt{a^2-x^2}}
      .\quad\text{)}
    \] %
  \item
    \( \displaystyle
        \int \sqrt{x^2 - a^2}\,dx
      = \frac{1}{2}
        \left( x\sqrt{x^2-a^2} 
               - a^2 \log \left|x + \sqrt{x^2-a^2}\right| \,\right).
    \) %
    \quad (ヒント: 上と同様に部分積分を行ない, 残った不定積分は %
    $t - x = \sqrt{x^2-a^2}$ なる変数変換によって処理する.)
    \qed
  \end{enumerate}
\end{question}


\begin{question}
  $a<b$ のとき, 変数変換 $t = \sqrt{\dfrac{x-a}{b-x}}$ を通して, 次が
  示される:
  \[
    \int_a^b \frac{dx}{\sqrt{x-a}\sqrt{b-x}}
    = 2 \int_0^\infty \frac{dx}{1+t^2}
    = 2 \left[\, \arctan t \,\right]_0^\infty
    = 2 \cdot \frac{\pi}{2}
    = \pi.
    \qed
  \]
\end{question}

\begin{question}
  $p > -1/2$, $q > -1/2$ に対して, 
  \[
    \int_0^{\pi/2} \sin^p \theta \cos^q \theta \,d\theta
    =
    \frac{1}{2} B\left( \frac{p+1}{2}, \frac{q+1}{2} \right).
  \qed
  \]
\end{question}

\noindent ヒント: $x^2 = \sin^2 \theta$ と変数変換せよ. (この公式の右
辺の $B$ はベータ函数である.)

\noindent 参考: この問題の結果を $q=0$, $p=n=0,1,2,\ldots$ と特殊化し,
前節の問題および参考で触れたガンマ函数とベータ函数の公式を使って,
Wallis の公式の別証明を与えることができる%
\footnote{例えば, 寺沢寛一著の『自然科学者のための数学概論[増訂版]』 
  (岩波書店)の第5章第21節[1]ではそのような取り扱い方をしている.}.

\begin{question}\label{q:int-exp-x2}
  Wallisの公式を利用して, %
  公式 $\displaystyle \int_{-\infty}^\infty e^{-x^2}\,dx = \sqrt{\pi}$ %
  を証明せよ.  \qed
\end{question}

\noindent ヒント: 『解析概論』の第35節例6 (p.117). 

\noindent 参考: この問題の結果は, 2変数函数の積分論を使うことが許され
れるならば, Wallisの公式を経由せずに直接証明される%
\footnote{『解析概論』の第94節の例2 (p.344).}.

\begin{question}[Hermite多項式の直交関係式]
  Hermite多項式 %
  \( \displaystyle
    H_n(x) = (-1)^n e^{x^2} \left( -\od{}{x} \right)^n e^{-x^2}
  \) %
  に関して, 次の直交関係式が成立する:
  \[
    \int_{-\infty}^\infty H_m(x) H_n(x) e^{-x^2}\,dx
    = \delta_{m,n} 2^n n! \sqrt{\pi}.
  \]
  ここで, $\delta_{m,n}$ は Kronecker のデルタ%
  \footnote{$\delta_{m,m} = 1$, $m\ne n$ のとき $\delta_{m,n} = 0$.}%
  である. \qed
\end{question}

\noindent ヒント: 『解析概論』の第3章の練習問題(13) (p.141).
公式 $\int_{-\infty}^\infty e^{-x^2}\,dx = \sqrt{\pi}$ を自由に用いて
良い.

%%%%%%%%%%%%%%%%%%%%%%%%%%%%%%%%%%%%%%%%%%%%%%%%%%%%%%%%%%%%%%%%%%%%%%%%%%%

\section{Taylorの定理の積分版}

微分のみを使った Taylor の定理の証明はかなり技巧的であった. $f^{(n)}$ %
連続性を仮定すると, 積分を用いて剰余項を表示する公式を証明できる. その
公式には, 部分積分を繰り返して帰納的に剰余項の形を決定してゆくという簡
明な証明が存在する.

$n$ は自然数であるとし, $f$ は開区間 $U$ 上の $C^n$ 級函
数であるとする. % 
$x, x + h \in U$ と $k = 1,2,\dots,n+1$ に対して, %
$x$, $h$ の函数 $R_k = R_k(x,h)$ を以下の式によって定義する:
\[
    f(x + h)
    = f(x)
    + \frac{1}{1!}     f'(x)        h
    + \frac{1}{2!}     f''(x)       h^2
    + \cdots
    + \frac{1}{(k-1)!} f^{(k-1)}(x) h^{k-1}
    + R_k(x,h).
\]

\begin{Theorem}[剰余項の積分表示]\label{Th:Taylor2}
  以上の状況のもとで, 剰余項 $R_n(x,h)$ は次のように表示される:
  \[
    R_n(x,h)
    = \int_0^1 \frac{(1-t)^{n-1}}{(n-1)!} f^{(n)}(x+th)h^n \,dt
    = \int_x^{x+h} \frac{x+h-\xi}{(n-1)!} f^{(n)}(\xi) \,d\xi.
  \]
\end{Theorem}

\begin{question}
  定理 \ref{Th:Taylor2}\ を証明せよ. \qed
\end{question}

\noindent ヒント: $n$ に関する帰納法で証明する. 高階も含めて導函数の連
続性を仮定したので, 導函数(に多項式をかけたもの)の積分を自由に行なえる. %
$\xi = x + th$ なる積分の変数変換によって, 中の式が右の式に変形できる
ことがわかるので, 中の式を示せば良い. $u(t)=f(x+th)$ に対する微分積分
法の基本公式より, $n=1$ の場合が成立することがわかる:
\[
  f(x+h) - f(x) = \int_0^1 f'(x + th) h \,dt.
\]%
$n$ まで成立したと仮定して $n+1$ の場合を示したい. そのためには, $t$ %
の函数 $u$, $v$ に関する部分積分の公式
\[
    \int_a^b u' v \,dt
  = \left[ u v \right]_a^b - \int_a^b u v' dt
\]%
を $u(t) = - \dfrac{(1-t)^n}{n!}$, $v(t) = f^{(n)}(x+th)h^n$ に対して
適用すれば良い.

\begin{question}
  剰余項の積分表示式を用いて %
  $\displaystyle \lim_{h\to0}\frac{|R_{n+1}(x,h)|}{|h|^n} = 0$ を証明
  せよ.  \qed
\end{question}

\noindent ヒント: $R_{n+1} = R_n - \frac{1}{n!}f^{(n)}(x)h^n$ の絶対値
を上から評価せよ. そのとき, %
$\int_0^1 \frac{(1-t)^{n-1}}{(n-1)!} \,dt = \frac{1}{n!}$ が成立するこ
とに注意せよ. 評価の途中で, 
不等式 $|\int_a^b g(t)\,dt| \le \int_a^b|g(t)|\,dt$ を用いよ. %

%%%%%%%%%%%%%%%%%%%%%%%%%%%%%%%%%%%%%%%%%%%%%%%%%%%%%%%%%%%%%%%%%%%%%%%%%%%
% 無限級数
%%%%%%%%%%%%%%%%%%%%%%%%%%%%%%%%%%%%%%%%%%%%%%%%%%%%%%%%%%%%%%%%%%%%%%%%%%%

\section{無限級数}

数列 $a_n$ に対してその無限和 $\sum\limits_{n=1}^\infty a_n$ を無限
級数もしくは単に級数と呼ぶ. 級数の値は極限 %
$\lim\limits_{N\to\infty}\sum\limits_{n=1}^N a_n$ が存在するとき, 
その値によって与えられる. 

全ての $a_n$ が非負の実数のとき, その級数は正項級数と呼ばれる. 正項級
数の極限値として $\infty$ を許す場合がよくあるが, 実数の中に $\infty$ 
のような数があるわけではないことには注意しなければいけない.

級数 $\sum\limits_{n=1}^\infty |a_n|$ が有限な値に収束するとき, 級数 %
$\sum\limits_{n=1}^\infty a_n$ も収束する. %
(このことは, 実数全体の集合の完備性(任意のCauchy列が収束すること)より
出る.) このとき, この級数は絶対収束すると言う. 絶対収束しないが, 収束
する級数は条件収束級数と呼ばれる. 各 $a_n$ が実数の範囲を越えて複素数
であったとしても, 同様に絶対収束級数が定義される.

正項級数と絶対収束級数の最も大事な性質は, その足し上げ方をどのように変
えても同一の極限値が得られることである. 例えば, 項を並べる順番を変えた
り, 部分級数の和に分割したり, …. また, 正項級数もしくは絶対収束級数が
2つあるとき, それらの積は有限級数と同じように分配律を用いて展開される.

以上で述べたことは, 『解析概論』の第43節(p.144)で詳しく解説されている. 
以下において, これらのことは自由に使ってよい.

\begin{question}[Dirichlet]
  $a_n$ は実数列であるとし, $\sum\limits_{n=1}^\infty a_n$ は条件収
  束級数であると仮定する. このとき, 無限和における項の順序を並び変える
  ことによって, 任意の値に収束させたり, $\infty$ や $-\infty$ に発散さ
  せたり, 収束性を失わせたりできることを示せ.
  \qed
\end{question}

\noindent ヒント: 『解析概論』の第43節(p.145).

\medskip『解析概論』の第44節(pp.148--152)では級数の絶対収束性の判定法
がいくつか解説されている. 基本的には, 有限な値に収束することがわかって
いる正項級数で, 絶対値の和が上から押さえられることを示すという方針で, 
判定法を作成するのである.

\begin{question}[Riemannのゼータ函数]
  $s>1$ のとき次の級数は収束し, $s\le1$ のとき発散する:
  \[
    \zeta(s) = \sum_{n=1}^\infty \frac{1}{n^s}.
  \]
  この $\zeta(s)$ は Riemann のゼータ函数と呼ばれている. \qed
\end{question}

\noindent 参考: Riemann のゼータ函数の詳しい性質を知りたければ, それを
実変数の函数ではなく, $s$ として複素数も許し複素函数とみなければいけな
い%
\footnote{$\zeta(s)$ は複素平面全体の上の有理型函数に解析接続される(極
  は $s=1$ のみ). 例えば, $\zeta(-1)=-\frac{1}{12}$ のような式が成立し
  ている. すなわち, 形式的には $1+2+3+\cdots=-\frac{1}{12}$ が成立して
  いる!! }. %
Riemann のゼータ函数の性質は, 素数の世界がどのようになっているかに密接
に関係している. 数論においてはこの他にも色々なゼータ函数が定義され
ていて, 大変興味深い世界を形作っている%
\footnote{Fermat 予想解決の最終ステップは, A.~Wiles による $\Q$ 上の半
  安定楕円曲線に関する谷山予想の解決であった. 谷山予想には同値な表現が
  いくつかあるが, そのうちの一つは, 「$\Q$ 上の楕円曲線のゼータ函数は
  函数等式をみたす」という形の命題である.}.

\begin{question}[Eulerの定数]
  次の極限が収束することを示せ:
  \[
    \gamma 
    = \lim_{n\to\infty} 
      \left( \frac{1}{1} + \frac{1}{2} + \dots +\frac{1}{n} - \log n \right).
  \]
  この定数 $\gamma$ は Euler 定数と呼ばれている. \qed
\end{question}

\begin{question}[Cauchyの凝集判定法]
  $a_n$ は非負実数からなる単調減少数列であるとする. %
  このとき, 正項級数 $\sum\limits_{n=1}^\infty a_n$ は %
  $\sum\limits_{n=1}^\infty 2^n a_{2^n}$ と同時に収束または発散する. 
  \qed
\end{question}

『解析概論』の第45節(pp.152--154)では絶対収束するとは限らない級数の収
束性の判定法がいくつか解説されてある. 条件収束する級数の収束の判定は, 
絶対収束する場合に比べて面倒である.

\begin{question}[交代級数]
  $a_1 \ge a_2 \ge a_3 \ge \cdots$ かつ %
  $a_n \to 0$ のとき, %
  級数 $\displaystyle\sum_{n=1}^\infty (-1)^{n-1} a_n$ は収束する. \qed
\end{question}

\begin{question}
  次が成立している:
  \[
    \frac{1}{1} - \frac{1}{3} + \frac{1}{5} - \frac{1}{7} + \cdots
    =
    \frac{\pi}{4},
    \qquad
    \frac{1}{1} - \frac{1}{2} + \frac{1}{3} - \frac{1}{4} + \cdots
    =
    \log 2.
  \] %
  これらの公式の左辺の級数は条件収束している. \qed
\end{question}

%%%%%%%%%%%%%%%%%%%%%%%%%%%%%%%%%%%%%%%%%%%%%%%%%%%%%%%%%%%%%%%%%%%%%%%%%%%

\section{Abelの級数変形法}\label{sec:Abel-partial-sum}

絶対収束するとは限らない級数の収束の判定には, Abelの判定法が便利である.

\begin{question}[Abelの判定法]\qstar{*}
  任意の数列 $v_n$ および非負の実数列 $\alpha_n$ を考え%
  \footnote{$v_n$ は複素数列でも良い. より一般には有限次元の実(もしく
    は複素)ベクトル空間内の点列でもよく, さらには 
    Banach 空間内の点列であってもよい. 要点は完備性である.}, %
  $s_n = v_1 + v_2 + \cdots + v_n$ と置き, %
  $\alpha_1 \ge \alpha_2 \ge \alpha_3 \ge \cdots$ と仮定する. このとき,
  以下が成立する:
  \begin{enumerate}
  \item 数列 $s_n$ が有界でかつ $\alpha_n \to 0$ ならば, %
    級数 $\sum\limits_{n=1}^\infty \alpha_n v_n$ は収束する.
  \item 数列 $s_n$ が収束するならば, ($\alpha_n \to 0$ でなくても) %
    級数 $\sum\limits_{n=1}^\infty \alpha_n v_n$ は収束する.
  \qed
  \end{enumerate}
\end{question}

\noindent ヒント: 次のAbelの級数変形を用いる. 簡単のために $s_0=0$ と
置くと, $1\le m \le n$ のとき,
\begin{align*}
  S_{m,n}
  & := \alpha_m v_m + \alpha_{m+1} v_{m+1} + \cdots + \alpha_n v_n
\\
  & =
  \alpha_m     (s_m     - s_{m-1}) +
  \alpha_{m-1} (s_{m-1} - s_{m-2}) +
  \cdots +
  \alpha_n     (s_n     - s_{n-1})
\\
  & =
                - \alpha_{m}    s_{m-1} +
  (\alpha_m     - \alpha_{m+1}) s_m     + 
  (\alpha_{m+1} - \alpha_{m+2}) s_{m+1} +
  \cdots +
  (\alpha_{n-1} - \alpha_{n}  ) s_{n-1} +
   \alpha_{n}                   s_n.
\end{align*}
より詳しい内容については, 『解析概論』の第45節の(VIII) (p.154)を見よ.
Abel の級数変形法は Abel の定理(問題 \qref{q:Abel-Th})の証明で使われる.

\noindent 注意: Abel の判定法において $v_{n}=(-1)^{n-1}$ と置けば, 交
代級数の収束判定法が導かれる. 

\begin{question}
  Abelの判定法を用いて次を示せ. %
  $\omega$ は複素数であり, $|\omega|=1$, $\omega\ne1$
  を満たしているする. このとき, %
  級数 $\sum\limits_{n=1}^\infty \dfrac{\omega^{n-1}}{n}$ は条件収束
  する.  \qed
\end{question}

Abelの級数変形の積分の場合における類似を考えると, 実はそれは部分積分に
他ならないことがわかる. 部分積分は不等式を出すための最も基本的な道具で
あり, Abelの級数変形法にもその一端が現われているのである.

\begin{question}[Abelの判定法の広義積分版]\qstar{*}
  $v$ は区間 $U=[a,b)$ 上の連続函数であり%
  \footnote{上と同様に $v$ は複素数値函数であっても良い.}, %
  $\alpha$ は $U$ 上の $C^1$ 級の実数値函数であるとする. %
  $s(x) = \int_a^x v(t)\,dt$ と置き, %
  $U$ 上で $\alpha \ge 0$, $\alpha' \le 0$ が成立している
  と仮定する. このとき, 以下が成立する.
  \begin{enumerate}
  \item $s(x)$ が $U$ 上で有界でかつ %
    $x\nearrow b$ のとき $\alpha(x) \to 0$ ならば, %
    広義積分 $\int_a^b \alpha(t)v(t)\,dt$ は収束する.
  \item $x\nearrow b$ で $s(x)$ がある値に収束するならば, %
    ($\alpha(x) \to 0$ でなくても) %
    広義積分 $\int_a^b \alpha(t)v(t)\,dt$ は収束する.
    \qed
  \end{enumerate}
\end{question}

\noindent ヒント: $s(x) = \int_a^x \alpha(t)v(t)\,dt$ と置く. このとき,
$a \le x \le y$ に対して,
\[
  S(y) - S(x)
  = \int_x^y \alpha(t) v(t) \,dt
  = \int_x^y \alpha(t) s'(t) \,dt
  = \left[ \alpha(t) v(t) \right]_x^y - \int_x^y \alpha'(t) s(t) \,dt.
\]
よって, $U$ 上で $|s(x)| \le R$ ならば, 
\begin{align*}
  |S(y) - S(x)|
  &
  \le |\alpha(y)s(y)| + |\alpha(x)s(x)|
  + \int_x^y |\alpha'(t)| |s(t)| \,dt
  \\
  &
  \le R \alpha(y) + R \alpha(x) + \int_x^y (- \alpha'(t)) R \,dt
  \\
  &
  = R \alpha(y) + R \alpha(x) - R (\alpha(y) - \alpha(x))
  = 2 R \alpha(x).
\end{align*}
(このように, 三角不等式や積分の中に絶対値を入れると大きくなるなどの基
本的な不等式を使って上からの評価を得るという計算方法は, 今までも何度も
出て来た手法である.)

上の問題の結果を使って以下を示せ. 

\begin{question}
  広義積分 $\displaystyle \int_2^\infty \frac{e^{ix}}{\log x} \,dx$ は
  条件収束する. 
  \qed
\end{question}

\begin{question}
  $f$ は正値微分可能函数であり, $x\to\pm\infty$ のとき $f(x)\to0$ であ
  るとし, ある $R\ge 0$ が存在して, $x\le -R$ ならば $f'(x)\ge 0$ であ
  り, $x\ge R$ ならば $f'(x) \le 0$ であると仮定する. %
  このとき, $p\ne 0$ なる任意の $p\in\R$ に対して, %
  広義積分 $\int_{-\infty}^\infty f(x) e^{-ipx}\,dx$ は収束する.
  \qed
\end{question}

\begin{question}
  広義積分 $\int_0^\infty e^{ix^2}\,dx$ は条件収束する. \qed
\end{question}

\noindent ヒント: $t=x^2$ と積分変数を置換し, Abelの判定法の広義積分版
を使う.

このように, 条件収束する級数や広義積分を見付けたければ, 級数の項もしく
は被積分函数が %
\(
  \text{(単調に減少しながら$0$に近付く因子)}
  \times
  \text{(よく振動する因子)}
\) %
の形のものを探せば良い.

%%%%%%%%%%%%%%%%%%%%%%%%%%%%%%%%%%%%%%%%%%%%%%%%%%%%%%%%%%%%%%%%%%%%%%%%%%%
% §. 極限の交換
%%%%%%%%%%%%%%%%%%%%%%%%%%%%%%%%%%%%%%%%%%%%%%%%%%%%%%%%%%%%%%%%%%%%%%%%%%%

\section{一様収束}

\begin{question}
  区間 $K=[0,1]$ 上の函数列 $f_n$ を $f_n(x)=x^n$ と定める. このとき, 
  各点 $x\in K$ において $f_n(x)$ は収束するが, $f_n$ は一様収束しない
  ことを直接の計算によって示せ. (連続函数列の一様収束極限で得
  られる函数が連続函数であることを使ってはいけない.) \qed
\end{question}

\begin{question}
  級数 %
  \( \displaystyle
    s(x)=\sum_{n=0}^{\infty} \frac{x^2}{(1+x^2)^n}
  \) %
  を考える. このとき, 級数 $s(x)$ は各点 $x\in\R$ において収束する
  が, $x=0$ を含む任意の区間上で一様収束していない. \qed
\end{question}

連続函数列が連続函数に各点収束しても, 一様収束するとは限らないことは次
の例よりわかる.

\begin{question}\label{q:not-uniform1}
  区間 $I=[0,\infty)$ 上の滑らかで有界な函数の列 $f_n$ を %
  $\varphi(x) = x e^{-x}$, $f_n(x) = n \varphi(nx)$ と定める. 
  このとき, 各点 $x\in I$ において %
  $f_n(x)$ は $0$ に収束するが, 函数列 $f_n$ は $0$ に一様収束しない.
  \qed
\end{question}

\noindent ヒント: 以上の3問は『解析概論』の第46節の例1 (p.155), 例2 %
(p.156), 例3 (p.157) にある.

級数の一様絶対収束性の判定のためには, 通常まず次の結果を使うことを考え
る.

\begin{question}[一様絶対収束の判定法]\label{q:unif-abs-conv-series}
  $f_n$ は区間 $U$ 上の函数列であるとし, $a_n$ は非負の実数列であると
  する. $U$ 上で $|f_n(x)| \le a_n$ であり, %
  正項級数 $\sum\limits_{n=1}^\infty a_n$ が有限な値に収束するならば,
  函数項級数 $\sum\limits_{n=1}^\infty f_n$ は一様に絶対収束する.
  \qed
\end{question}

\noindent 注意: 広義積分でも同様な結果が成立する
(問題 \qref{q:unif-abs-conv-int}).

\begin{question}
  任意の $a>1$ に対して, %
  $\displaystyle \zeta(s)=\sum\limits_{n=1}^\infty \frac{1}{n^s}$
  は $s\in[a,\infty)$ に関して一様に絶対収束する. このことより, %
  $\zeta(s)$ は $s>1$ で連続であることがわかる.  \qed
\end{question}

以下, 閉区間 $K=[a,b]$ 上の連続函数 $f$ に対して,
\[
  ||f|| := \sup_{x\in K} |f(x)|
\]
と置く. $|f|$ は $K$ 上の連続函数なので $K$ 上最大値を持つ. よって, 
$||f||$ は well-defined である. $||\cdot||$ は sup norm (スープ・ノル
ム)と呼ばれる. 

\begin{question}
  $f$ および $f_1,f_2,f_3,\ldots$ が閉区間 $K=[a,b]$ 上の連続函数であ
  るとき, 函数列 $f_n$ が $f$ に $K$ 上一様収束するための必要十分条件
  は, $||f_n - f|| \to 0$ が成立することである. \qed
\end{question}

\begin{question}
  $f$, $g$ が閉区間 $K=[a,b]$ 上の連続函数であり, $\alpha\in\R$ のとき,
  以下が成立する:
  \begin{enumerate}
  \item $||f||=0$ ならば $f$ は恒等的に $0$.
  \item $||\alpha f|| = |\alpha|\,||f||$.
  \item $||f+g|| \le ||f|| + ||g||$. 
    \qed
  \end{enumerate}
\end{question}

\noindent このように, $||\cdot||$ は「絶対値もどき」の性質を有している.

\begin{question}\label{q:C0}\qstar{*}
  閉区間 $K = [a,b]$ 上の連続函数列 $f_n$ に対して, $f_n$ が $K$ 上の
  ある連続函数に一様収束するための必要十分条件は次が成立することである:
  \begin{description}
  \item[$(\ast)$] 任意の $\varepsilon > 0$ に対して, ある $N$ が存
    在して, $m,n \ge N$ ならば $||f_n -f_m|| \le \varepsilon$ が成立す
    る.  \qed
  \end{description}
\end{question}

\noindent ヒント: 必要性は簡単に示せる. 十分性は以下のようにして示せば
良い. 任意の $\varepsilon > 0$ を固定する. $(\ast)$ を仮定すると, %
ある $N$ が存在して, %
$m,n\ge N$ ならば $||f_n - f_m|| \le \varepsilon$. %
よって, そのとき, 任意の $x\in K$ に対して, %
$|f_n(x) - f_m(x)| \le \varepsilon$. %
これで, 各 $x\in K$ に対して, 数列 $f_n(x)$ は Cauchy 列であることがわ
かった. よって, 数列 $f_n(x)$ は収束する. %
その収束先を $f(x)$ と書くことにする. %
$m \to \infty$ とすることによって, %
$n \ge N$ ならば, 任意の $x\in K$ に対して, %
$|f_n(x) - f(x)| \le \varepsilon$ となることがわかる. %
これで, $f_n$ が $f$ に $K$ 上一様に収束することがわかった. %
以下, $n \ge N$, $x \in K$ とする. %
各 $f_n$ は連続なので, ある $\delta > 0$ が存在して, %
$y\in K$ かつ $|y - x| < \delta$ ならば %
$|f_n(y) - f_n(x)| \le \varepsilon$ が成立する. %
このとき,
\[
  |f(y) - f(x)|
  \le |f(y) - f_n(y)| + |f_n(y) - f_n(x)| + |f_n(x) - f(x)|
  \le 3 \varepsilon.
\]
これで, $f$ の連続性も示せた. (この論法を $3\varepsilon$ 論法と呼ぶこ
ともある.)

\noindent 注意: 上の証明をよく読めば, 連続函数の列 $f_n$ が函数 $f$ に
一様収束するとき, $f$ もまた連続函数になることもわかる.

\medskip
\begin{small}

\noindent 参考: %
$C(K) = \{ \text{閉区間 $K$ 上の実数値連続函数全体} \}$ %
と置く. 上の問題の結果は, $C(K)$ における sup norm $||\cdot||$ に関す
る Cauchy 列が常に(強)収束することを意味している. %
一般に, ノルム $||\cdot||$ に関する Cauchy 列が常に収束するような空間
は完備であると言い, そのような空間は Banach 空間と呼ばれる. すなわち, 
上の問題によって, $(C(K), ||\cdot||)$ は Banach 空間であることがわかっ
た. このように函数の空間を系統的に扱うための一般論として, 函数解析学が
整備されている.

\end{small}

\begin{question}[Weierstrassの多項式近似定理]\qstar{*}
  閉区間 $K=[a,b]$ 上の任意の連続函数 $f$ に対して, $f$ に $K$ 上一様
  収束するような多項式函数の列が存在する. \qed
\end{question}

\noindent ヒント: 『解析概論』の第78節 (p.284)に初等的な証明が書いてある.

%%%%%%%%%%%%%%%%%%%%%%%%%%%%%%%%%%%%%%%%%%%%%%%%%%%%%%%%%%%%%%%%%%%%%%%%%%%

\section{函数列の積分}

\begin{question}
  開区間 $U$ 上の連続函数の列 $f_n$ が函数 $f$ に一様収束していると仮
  定する. このとき, $x,c\in U$ に対して, %
  $F_n(x) = \int_c^x f_n(t)\,dt$, %
  $F(x) = \int_c^x f(t) \,dt$ %
  と置くと, $U$ 上の函数列 $F_n$ は $F$ に一様収束する.
  \qed
\end{question}

\begin{small}

  \noindent 参考: Lebesgue 式積分論ではもっと強力で使い易い結果を示す
  ことができる. Lebesgue 式積分論では測度空間 $X$ 上の可測函数%
  \footnote{ここで, 測度空間やその上の可測空間の定義はできない. しかし,
    もちろん, $\R^n$ は Lebesgue 測度とともに測度空間であり, その上の
    連続函数および連続函数から各点収束極限を何度か取って得られる函数は
    すべて可測函数である.}%
  $f$ が $\int_X|f(x)|\,dx < \infty$ をみたすとき, $f$ は可積分もしく
  は積分可能(integrable)であると言われ, 積分 $\int_X f(x)\,dx$ が定義さ
  れる. %
  以下 $f_n$ が $X$ 上の可積分函数の列であり, ある函数にほとんどいたる
  ところ%
  \footnote{「ほとんどいたるところ」と数学らしくない表現が使われている
    が, Lebesgue 式積分論において, それは「測度零のある可測集合の外では」
    という意味であると厳密に定義される.}%
  収束していると仮定する. このとき, 以下が成立する.
  %
  \begin{Theorem}[単調収束定理]
    各 $f_n$ が実数値函数であり, $f_1 \le f_2 \le f_3 \le \cdots$ をみ
    たしているならば, 両辺の値として $\infty$ も許せば, 
    \(%
      \lim\limits_{n\to\infty} \int_X f_n(x) \,dx
      =
      \int_X \lim\limits_{n\to\infty} f_n(x) \,dx
    \)%
    が常に成立する.
  \end{Theorem}
  %
  \begin{Theorem}[Lebesgueの収束定理]
    $n$ によらない可積分実数値函数 $g$ が存在して, $|f_n| \le g$
    をみたしているならば, $\lim f_n$ も可積分であり,
    \(%
      \lim\limits_{n\to\infty} \int_X f_n(x) \,dx
      =
      \int_X \lim\limits_{n\to\infty} f_n(x) \,dx
    \)%
    が成立する.
  \end{Theorem}
  %
  要するに, 単調に収束する函数列や, 収束する函数列で積分が絶対収束する
  函数で一様に押さえられているようなものに関しては, 積分と $\lim$ の交
  換を自由に行なってよいのである. これらの定理を使うと, 一様収束の判定
  という, 少々面倒な手続きを回避できる%
  \footnote{ただし, これらの便利な定理をこの演習の解答において用いては
    いけない. しかし, この演習のように, 初等微分積分学の基礎を修得する
    というような特別な目的がない限り, このように便利な定理は証明抜きで
    認めて自由に使って良い. }.

\end{small}

\begin{question}
  問題 \qref{q:not-uniform1}\ に現われた区間 $I=[0,\infty)$ 上の函数列 %
  $f_n(x) = n^2 x e^{-nx}$ を考える. このとき, 各点 $x\in I$ で $f_n(x)$ %
  は $0$ に収束し, 各 $f_n$ の $I$ 上での広義積分は絶対収束するにもか
  かわらず,
  \(%
    \lim\limits_{n\to\infty} \int_0^\infty f_n(x) \,dx
    \ne
    0
  \)%
  が成立することを示せ. \qed
\end{question}

\begin{small}
  \noindent この問題の函数列は, 単調収束定理もLebesgueの収束定理も適用
  することができない例にもなっている.
\end{small}

%%%%%%%%%%%%%%%%%%%%%%%%%%%%%%%%%%%%%%%%%%%%%%%%%%%%%%%%%%%%%%%%%%%%%%%%%%%

\section{函数列の微分}

微分可能函数の列の極限によって得られる函数の微分可能性については, 次の
結果が基本的である.

\begin{question}\label{q:diff-seq}\qstar{*}
  $f_n$ が区間 $U$ 上の微分可能函数の列であり, $U$ 上のある函数 $f$ に
  収束しているとする. このとき, 導函数の列 $f'_n$ が $U$ 上のある連続
  函数 $g$ に一様収束しているならば, $f$ は微分可能であり, $f'=g$ が成
  立する.
\end{question}

\noindent ヒント: 『解析概論』の第47節の({\bf C}${}'$)の級数に対する証
明(p.160)を, 函数列の場合に焼き直せば良い.

問題 \qref{q:C0}\ の類似の結果を $C^1$ 級函数に対して与えよう. %
閉区間 $K=[a,b]$ に対して,
\[%
  C^1(K)
  = \{\, f : K \to \R \mid \text{$f$ は $C^1$ 級である} \}
\] %
と置き, $f \in C^1(K)$ に対して, 
\[
  ||f||_1 = \max\{\, \sup_{x\in K}|f(x)|,\, \sup_{x\in K} |f'(x)| \,\}
\]
と置く. この $||\cdot||_1$ を $C^1$-norm と呼ぶことにする.

\begin{question}
  $f,g\in C^1(K)$, $\alpha\in\R$ に対して以下が成立する:
  \begin{enumerate}
  \item $||f||_1=0$ ならば $f$ は恒等的に $0$.
  \item $||\alpha f||_1 = |\alpha|\,||f||_1$.
  \item $||f+g||_1 \le ||f||_1 + ||g||_1$. 
    \qed
  \end{enumerate}
\end{question}

\begin{question}\label{q:C1}\qstar{*}
  $K$ 上の $C^1$ 級函数の列 $f_n$ が次の条件を満たしていると仮定する:
  \begin{description}
  \item[$(\ast)$] 任意の $\varepsilon > 0$ に対して, ある $N$ が存
    在して, $m,n \ge N$ ならば $||f_n -f_m||_1 \le \varepsilon$ が成立す
    る. 
  \end{description}
  このとき, $||f_n - f||_1 \to 0$ を満たすような $f \in C^1(K)$ が唯一存
  在する.  \qed
\end{question}

\noindent ヒント: 問題 \qref{q:C0}\ の結果より, $f_n$, $f'_n$ はそれぞ
れ $K$ 上のある連続函数 $f$, $g$ に一様収束することがわかる. あとは, %
$f$ が微分可能で $f'=g$ となることを示せば十分である. %
$\varepsilon > 0$ と $x\in K$ を任意に固定する. %
$h$ は $x+h\in K$ を満たす範囲を動くことにし, %
$R=|f(x+h)-f(x)-g(x)h|$, $A=|f(x+h)-f(x)-(f_n(x+h)-f_n(x))|$, %
$B=|-g(x)+f_n(x)|\,|h|$, $C=|f_n(x+h)-f_n(x)-f'_n(x)h|$ と置くと, 三角
不等式より, $R \le A + B + C$. $f'_n$ は $g$ に一様収束するので, ある 
$N$ が存在して, $n\ge N$ ならば $|f'_n - g| \le \varepsilon$. %
そのとき, $B \le \varepsilon |h|$.  %
$f_n$ は微分可能なので, ある $\delta > 0$ が存在して, %
$|h|\le\delta$ ならば $C \le \varepsilon |h|$. %
問題は $A$ の評価である. %
$f_m - f_n$ に平均値の定理を用いることによって, $m,n \ge N$ に対して, 
ある $0 < \theta < 1$ なる $\theta$ が存在して, $\xi=x+\theta h$ と置
くと,
\begin{align*}
  & |f_m(x+h) -f_m(x) - (f_n(x+h) - f_n(x))|
  = |(f'_m(\xi) - f'_n(\xi)) h| 
  \\
  & \le ( |f'_m(\xi) - g(\xi)| + |g(\xi) - f'_n(\xi)|)|h|
  \le 2 \varepsilon |h|.
\end{align*}
よって, $m\to\infty$ として, %
$n\ge N$ ならば $A \le 2 \varepsilon |h|$ が成立することがわかる. %
以上をまとめると, $|h| \le \delta$, $n\ge N$ ならば %
$R \le 4 \varepsilon |h|$ が成立することがわかる.

\medskip
\begin{small}

\noindent 参考: 上の結果は $C^1(K)$ が $C^1$-norm に関して完備であるこ
とを意味している. すなわち, $(C^1(K), ||\cdot||_1)$ は Banach 空間である.

\end{small}

%%%%%%%%%%%%%%%%%%%%%%%%%%%%%%%%%%%%%%%%%%%%%%%%%%%%%%%%%%%%%%%%%%%%%%%%%%%

\section{巾級数}

$\sum\limits_{n=0}^\infty a_n x^n$ の形の級数を $x$ の巾級数と呼ぶ. 以
下, $a_n$ は複素数でも良いことにするが, 特別に断らない限り $x$ として
実数のみを考える. (しかし, 全ての結果は $x$ として複素数を考えても成立
している.)

\begin{question}
  巾級数 $f(x) = \sum\limits_{n=0}^\infty a_n x^n$ がある $x = x_0$ に
  おいて絶対収束しているならば, $|x| \le |x_0|$ において一様絶対収束し
  ている.  よって, $f(x)$ は $|x| \le |x_0|$ における連続函数を与える.
  \qed
\end{question}

\noindent ヒント: 前半は『解析概論』の定理47 (p.181)よりも少し弱い形で
あり, 問題 \qref{q:unif-abs-conv-series}\ の結果を使えば簡単に証明でき
る.

\begin{question}[Cauchy-Hadamard]
  巾級数の収束半径の定義を説明し, 巾級数 $f(x) = \sum a_n x^n$ の収束
  半径 $r$ が次のような表示を持つことを証明せよ:
  \[
    \frac{1}{r} = \limsup_{n\to\infty} \sqrt[n]{|a_n|}.
  \qed
  \]
\end{question}

\noindent ヒント: 『解析概論』の定理48 (p.182).

\begin{question}
  巾級数 $\sum \frac{n!}{(2n)!} x^n$ の収束半径は $\infty$ である. 
  \qed
\end{question}

\begin{question}
  巾級数 $\sum n! x^n$ の収束半径は $0$ である. \qed
\end{question}

\begin{question}
  $0$ でない多項式函数 $p(x)$ に対して, 巾級数 $\sum p(n) x^n$ の収束
  半径は $1$ である. \qed
\end{question}

\begin{question}[超幾何級数]\label{q:hypergeom-series}
  超幾何級数と呼ばれる級数とは次の巾級数のことである:
  \[
    F(\alpha, \beta, \gamma; x)
    =
    \sum_{n=0}^\infty
    \frac{
      \alpha (\alpha + 1) \cdots (\alpha + n - 1)
      \cdot
      \beta (\beta + 1) \cdots (\beta + n - 1)
      }{
      \gamma (\gamma + 1) \cdots (\gamma + n -1)
      \cdot
      n!
      }
    x^n.
  \]%
  超幾何級数が well-defined でかつ無限級数になるような %
  $\alpha$, $\beta$, $\gamma$ に対して, 超幾何級数の収束半径は $1$ に
  等しい.  \qed
\end{question}

\noindent ヒント: 『解析概論』の第52節の例2 (p.188).

\noindent 参考: 実はこの級数の値は, $\Repart \alpha > 0$, %
$\Impart(\gamma-\alpha) > 0$, %
$\gamma \ne 0,-1,-2, \dots$, $|x|<1$ のとき, 問題 %
\qref{q:hypergeom-int}\ で定義された超幾何積分に等しい.

\begin{question}
  $|x|<1$ において, $x F(1,1,2;x) = - \log(1 - x)$. \qed
\end{question}

\begin{question}
  $|x|<1$ において, $F(\alpha,\gamma,\gamma; x) = (1 - x)^{-\alpha}$.
  \qed
\end{question}

\begin{question}
  $|x|<1$ において, %
  $\lim\limits_{\beta\to\infty}F(1,\beta,1; x/\beta) = e^x$.
  \qed
\end{question}

\begin{question}[超幾何微分方程式]\label{q:hypergeom-eq}
  超幾何級数 $F(\alpha,\beta,\gamma;x)$ は次の線型常微分方程式を満たし
  ている:
  \[
    \left[
      x(1 - x) \frac{d^2}{dx^2}
      + (\gamma - (\alpha + \beta + 1) x) \frac{d}{dx}
      - \alpha\beta
    \right] F(\alpha,\beta,\gamma; x) = 0.
  \]%
  この微分方程式を(Gaussの)超幾何微分方程式%
  \footnote{Gaussの超幾何微分方程式は, 自明でない確定特異点型線型微分
    方程式系の中で最も簡単な例である. 現在においては, 偏微分方程式の場
    合も含めて, 確定特異点型線型微分方程式系の一般的理論は, $D$ 加群の
    理論として, 代数幾何的な言葉によって整理されている.}%
  と呼ぶ. \qed
\end{question}

\begin{question}
  巾級数 $\sum a_n x^n$ の収束半径は $r$ であるとし, $p(x)$ は $0$ で
  ない任意の多項式函数であるとする. %
  このとき, 巾級数 $\sum p(n)a_n x^n$ の収束半径は $r$ である. 
  \qed
\end{question}

\begin{question}
  巾級数 $f(x) = \sum a_n x^n$ の収束半径は $r$ であるとする. このとき,
  $f(x)$ は $|x| < r$ における微分可能函数を与え, 次を満たしている:
  \[
    f'(x) = \sum_{n=1}^\infty a_n n x^{n-1}. \qed
  \]
\end{question}

\noindent ヒント: 一つ上の問題の結果を $p(n)=n$ の場合に用い, 函数列の
微分に関する基本的な結果(問題 \qref{q:diff-seq})を使う.  『解析概論』
の定理49 (p.183)の解説の中で別の直接的な証明が紹介されている.

\begin{question}[Abelの定理]\label{q:Abel-Th}\qstar{*}
  巾級数 $\sum a_n x^n$ の収束半径は $1$ であり, %
  級数 $a_0 + a_1 + a_2 + \cdots$ は収束していると仮定する.  %
  このとき, 巾級数 $f(x) = \sum a_n x^n$ は閉区間 $[0,1]$ 上で一様
  収束している. よって, $f(x)$ は閉区間 $[0,1]$ 上の連続函数を与える. 
  特に, 次が成立する:
  \[
    \lim_{x\nearrow 1} \sum_{n=0}^\infty a_n x^n
    =
    a_0 + a_1 + a_2 + \cdots.
  \qed
  \]
\end{question}

\noindent ヒント: 『解析概論』の定理50 (p.184). Abelの級数変形法を用い
る. Abelの級数変形法の節 \ref{sec:Abel-partial-sum}\ の記号を使えば, %
$\alpha_n = x^n$ ($x\in[0,1]$), $v_n = a_n$ にAbelの級数変形法を用
いるということになる.  %
$s_n = a_0 + a_1 + \dots + a_n$ と置く. このとき, $s_n$ %
は収束するので, 任意の $\varepsilon > 0$ に対して, %
ある $N$ があって, %
$m,n \ge N$ ならば, $|s_n - s_m| \le \varepsilon$ が成立する.  %
$n \ge N$ に対して, $\sigma_n = s_n - s_N$ と置くと, %
$|\sigma_n| \le \varepsilon$ である. %
$x\in[0,1]$ に対して, $f_n(x) = \sum\limits_{k=0}^n a_k x^k$ と置
く. このとき, $N \le m \le n$ ならば, 任意の $x\in[0,1]$ に対して, %
\begin{align*}
  |f_n(x) - f_m(x)|
  & =
  |a_{m+1} x^{m+1} + \dots + a_n x^n |
  \\
  & =
  |(\sigma_{m+2} - \sigma_{m+1}) x^{m+1}
  + \dots + (\sigma_n - \sigma_{n-1}) x^n|
  \\
  & =
  |- \sigma_{m+1}x^{m+1} + \sigma_{m+1}(x^{m+1} - x^{m+2})
  + \dots + \sigma_{n-1}(x^{n-1} - x^n) + \sigma_n x^n|
  \\
  & \le
  \varepsilon x^{m+1} + \varepsilon (x^{m+1} - x^{m+2})
  + \dots + \varepsilon (x^{n-1} - x^n) + \varepsilon x^n
  \\
  & =
  2 \varepsilon x^{m+1}
  \le 2 \varepsilon.
\end{align*}
これで, $f_n$ が閉区間 $[0,1]$ 上で一様収束することがわかった.

\begin{question}\qstar{*}
  $a > 0$, $b > 0$ のとき, 
  \[
    \int_0^1 \frac{x^{a-1}}{1+x^b} \,dx
    =
    \sum_{n=0}^\infty \frac{(-1)^n}{a+nb}
    =
    \frac{1}{a} - \frac{1}{a+b} + \frac{1}{a+2b} - \cdots.
  \]
  例えば,
  \begin{align*}
    &
    a=1, b=1 \quad \text{とすれば,}
    \qquad
    \frac{1}{1} - \frac{1}{2} + \frac{1}{3} - \cdots = \log2,
    \\
    &
    a=1, b=2 \quad \text{とすれば,}
    \qquad
    \frac{1}{1} - \frac{1}{3} + \frac{1}{5} - \cdots = \frac{\pi}{4}.
  \qed
  \end{align*}
\end{question}

\noindent ヒント: これは『解析概論』の第4章の練習問題(12) (p.200)とその
注意そのものである. これを示すためには, $0 \le x \le 1$ で成立する式
\[
  \frac{x^{a-1}}{1+x^b}
  =
  \sum_{n=0}^\infty (-1)^n x^{a+nb-1}
\]%
を項別積分できることを示せば良い. 
そのとき, Abel の定理を使うと簡単である.

%%%%%%%%%%%%%%%%%%%%%%%%%%%%%%%%%%%%%%%%%%%%%%%%%%%%%%%%%%%%%%%%%%%%%%%%%%%

\section{積分によって表示された函数}

この節では二変数函数 $f(x,y)$ を $y$ に関して積分して得られる $x$ の函
数 $F(x) = \int_a^b f(x,y)\,dy$ を扱う. 
まず, 二変数函数に関する基本的な結果をまとめておこう.

\begin{Definition}[連続性]
  $f$ は開区間の直積 $U=(a,b)\times(c,d)$ 上の函数であるとする. $f$ が
  点 $(x_0,y_0)\in U$ において連続であるとは, 任意の $\varepsilon > 0$%
  に対して, ある $\delta > 0$ が存在して, $(x,y)\in U$ かつ %
  $|x -x_0|,|y-y_0| \le \delta$ ならば %
  $|f(x,y)-f(x_0,y_0)| \le \varepsilon$ が成立することである. $U$ の任
  意の点で $f$ が連続なとき, $f$ は $U$ 上の連続函数であると言う.
\end{Definition}

\begin{Theorem}[Bolzano-Weierstrass]
  $\R^2$ 内の任意の有界点列は収束する部分列を持つ.
\end{Theorem}

\begin{Theorem}[Heine-Borel]
  閉区間の直積 $K=[a,b]\times[c,d]$ が開区間の直積の族 %
  $U_i=(a_i,b_i)\times(c_i,d_i)$ ($i\in I$) で覆われていると仮定する:%
  $K \subseteq \bigcup\limits_{i\in I} U_i$. このとき, 有限個の %
  $i_1,\ldots,i_n$ が存在して, %
  $K \subseteq U_{i_1}\cup\dots\cup U_{i_n}$ が成立する.
\end{Theorem}

これらの定理を使うと, 一変数の場合と全く同様にして以下の結果を示すこと
ができる.

\begin{Theorem}
  閉区間の直積 $K=[a,b]\times[c,d]$ 上の任意の実数値連続函数は, $K$ に
  おいて最大値と最小値を取る.
\end{Theorem}

\begin{Theorem}
  閉区間の直積 $K=[a,b]\times[c,d]$ 上の任意の連続函数は $K$ 上一様連
  続である. すなわち, 任意の $\varepsilon > 0$ に対して, %
  ある $\delta > 0$ が存在して, %
  $(x_1,y_1),(x_2,y_2) \in U$ かつ %
  $|x_2 - x_1|,|y_2 - y_1| \le \delta$ ならば %
  $|f(x_2,y_2)-f(x_1,y_1)| \le \varepsilon$ が成立する.
\end{Theorem}

\noindent 注意: ここでは, 2次元の場合について述べたが, 一般の高次元の
場合でも全く同様のことが成立している. また, 閉集合や開集合などの言葉を
避けるために, 閉区間や開集合の直積に関する結果について述べた.  それで
も定理の本質的な部分は失われないのだが, すっきりした理解をするためには, 
やはり位相空間の言葉を用いるべきであろう.

\begin{question}[積分表示された函数の連続性]
  $f(x,y)$ が開区間 $U$ と閉区間 $K=[a,b]$ の直積 $U \times K$ の上の
  連続函数であるとき, $F(x)=\int_a^b f(x,y)\,dy$ によって定義される %
  $U$ 上の函数 $F$ は連続である.
\end{question}

\noindent ヒント: $U$ に含まれる任意の閉区間 $J$ の上で連続であること
を示せば良い. $x,x_0\in J$ に対して,
\[
  |F(x) - F(x_0)|
  \le \int_a^b |f(x,y) - f(x_0,y)| \,dy.
\]
$f(x,y)$ は $J\times K$ の上で一様連続であるから, %
任意の $\varepsilon > 0$ に対して, ある $\delta > 0$ が存在して, %
$|x - x_0| \le \delta$, $y\in K$ ならば %
$|f(x,y) - f(x_0,y)| \le \varepsilon$ が成立する. そのとき, %
\break
$|F(x) - F(x_0)| \le |b-a| \varepsilon$.  

\begin{Definition}[偏導函数]
  $f(x,y)$ は開区間の直積 $U=(a,b)\times(c,d)$ 上の函数であるとする. 
  点 $(x,y)\in U$ において, $f$ が $x$ に関して偏微分可能であると
  は, 次の極限が存在することである:
  \[
    f_x(x,y) = \pd{f}{x}(x,y)
    = \lim_{h\to0} \frac{f(x+h,y) - f(x,y)}{h}.
  \] %
  $f$ が $U$ 上の任意の点において $x$ に関して偏微分可能であるとき, %
  $f$ は $x$ に関して $U$ 上偏微分可能であると言い, 上の式で定義される%
  $U$ 上の函数$f_x$ を $f$ の $x$ に関する偏導函数と呼ぶ.
  ($y$ に関する偏微分可能性と偏導函数も同様に定義される.)
\end{Definition}

\begin{question}[積分と微分の交換]
  $f(x,y)$ は開区間 $U$ と閉区間 $K=[a,b]$ の直積 $U\times K$ 上の連続
  函数であり, $f$ は $x$ に関して偏微分可能であり, $x$ に関する偏導函
  数 $f_x$ は $U\times K$ 上連続であると仮定する. このとき, 
  $U$ 上の函数 $F$ を $F(x) = \int_a^b f(x,y)\,dy$ によって定義すると,
  $F$ は微分可能であり, $F'(x) = \int_a^b f_x(x,y)\,dy$ が成立してい
  る.
\end{question}

\noindent ヒント: 平均値の定理より, $0 < \theta(x,y) < 1$ をみたす %
$U\times K$ 上の函数 $\theta$ が存在して, %
\break
$f(x+h,y) - f(x,y) = f(x+\theta(x,y)h,y)h$ が成立する. よって,
\[
  R := F(x+h) - F(x) - \int_a^b f_x(x,y)\,dy \cdot h
  = \int_a^b ( f_x(x+\theta(x,y)h,y) - f_x(x,y))\,dy \cdot h.
\]
ここで, $f_x$ が $U$ に含まれる任意の閉区間と $K$ の直積上で一様連続で
あることを使うと, 任意の $\varepsilon > 0$ と $x\in U$ に対して, %
ある $\delta > 0$ が存在して, $|h|\le\delta$, $y\in K$ ならば %
\break
$|f_x(x+\theta(x,y)h,y) - f_x(x,y)| \le \varepsilon$ が成立することが
わかる. このとき, $|R| \le (b-a) \varepsilon |h|$.

\begin{question}[二重積分の順序の交換]
  $f(x,y)$ は閉区間の直積 $J\times K = [a,b]\times[c,d]$ 上の連続函数
  であるとする. $n=1,2,3,\ldots$, $i=1,2,\dots,n$ に対して, $y_{n,i}$ %
  を $y_{n,i} \in [c+\frac{d-c}{n}(i-1), c+\frac{d-c}{n}i]$ をみたすよ
  うに選んでおく. このとき, $J=[a,b]$ 上の函数列 $F_n$ を %
  \( \displaystyle
    F_n(x) = \sum_{i=1}^{n} f(x, y_{n,i}) \,\frac{d-c}{n}
  \) %
  と定めると, 各 $F_n$ は連続であり, %
  $J$ 上 $F(x) = \int_c^d f(x,y) \,dy$ に一様収束する. よって, 次が成
  立することがわかる:
  \[
    \int_a^b \left( \int_c^d f(x,y) \,dy \right) \,dx
    =
    \int_c^d \left( \int_a^b f(x,y) \,dx \right) \,dy.
  \qed
  \]
\end{question}

\noindent ヒント: 『解析概論』の定理41 (p.162の定理)の(B).

\begin{small}
  \noindent 参考: Lebesgue 積分論における Fubini の定理はこの問題の結
  果のすっきりしたわかりやすい一般化を与えている. それによると, $f$ の
  連続であることや積分の範囲が閉区間であることは本質的ではないことがわ
  かる. Fubini の定理が成立するためには, 積分の範囲はどのようなもので
  も良く, $f$ が可積分であれば十分なのである.
\end{small}

%%%%%%%%%%%%%%%%%%%%%%%%%%%%%%%%%%%%%%%%%%%%%%%%%%%%%%%%%%%%%%%%%%%%%%%%%%%

\section{広義積分によって表示された函数}

さて, 前節においては閉区間の上の積分によって表示された函数の連続性や微
分可能性について調べた. しかし, 例えば, 重要な例であるガンマ函数は広義
積分によって定義されている. したがって, 広義積分によって表示された函数
の性質を調べる方法も知っておく必要がある.

$f(x,y)$ は開区間 $U$ と区間 $I=[a,b)$ ($b$ の値として $\infty$ も許す)
の直積 $U\times I$ 上の連続函数であるとする. $U$ 上の函数 $F$ を
\[
  F(x)
  = \int_a^b f(x,y)\,dy
  = \lim_{\beta\nearrow b} \int_a^\beta f(x,y)\,dy
\]
によって定義する. ただし, 右辺の広義積分は常に収束しているものと仮定す
る. このように表示された函数 $F$ の連続性や微分可能性はどのようにして
調べたら良いのであろうか? 今まで得られた結果を使って調べるためには, 二
段階に分けて考えなければいけない.
\begin{enumerate}
\item $\beta \in [a,b)$ なる $\beta$ に対して, 次の函数を調べる:
  \[%
    F(\beta,x) = \int_a^\beta f(x,y)\,dy.
  \]%
\item 次に, $\beta\nearrow b$ における $F(\beta,x)$ の $F(x)$ への収束
  の仕方を調べる. 例えば, $x$ に関して一様に $F(\beta,x)$ が $F(x)$ に
  収束してないか, もしくは, $x$ に関する偏導函数 $F_x(\beta,x)$ が一様
  収束してないかなどについて調べる.
\end{enumerate}

絶対収束の範囲では次の結果が便利である.

\begin{question}[一様絶対収束の判定法]\label{q:unif-abs-conv-int}
  $f(x,y)$ は開区間 $U$ と区間 $I=[a,b)$ ($b$ の値として $\infty$ も許
  す)の直積 $U\times I$ 上の連続函数であるとする. $x$ によらない $I$ %
  上の実数値連続函数 $g(y)$ で, %
  $g \ge 0$ でかつ広義積分 $\int_a^b g(y)\,dy$ が絶対収束し, %
  $U\times I$ 上で $|f(x,y)| \le g(y)$ を満たすものが存在すると
  仮定する. %
  このとき, 広義積分 $F(x) = \int_a^b f(x,y) \,dy$ は $x$ に関して一様
  に絶対収束し, $U$ 上の連続函数を与える. \qed
\end{question}

\noindent ヒント: 『解析概論』の p.167 の注意.

\noindent 注意: この問題と問題 \qref{q:unif-abs-conv-series}\ を比べて
見よ. どちらも, 絶対値を上から一様に押さえるという形の一様絶対収束判定
法になっている. このような考え方は絶対収束の判定において基本的である.

\noindent 注意: 一様絶対収束性の証明において, $f$ や $g$ が連続函数で
あるという仮定は必要ではない. Lebesgue 積分論を知らない人のために, 余
計な仮定を付けておいたのである. (もちろん, $F(x)$ の連続性を示すために
は, 連続性の条件を仮定に入れておかねばならない.)

この結果を利用して, 以下を示せ.

\begin{question}[Fourier変換の定義]
  $f$ は $\R$ 上の連続函数であり, %
  広義積分 $\int_{-\infty}^\infty f(x)\,dx$ は絶対収束していると仮定す
  る. このとき, $p\in\R$ に関して, 次の広義積分は一様に絶対収束する:
  \[
     \hat{f}(p) = \int_{-\infty}^\infty f(x) e^{-ipx} \,dx.
  \]%
  ($\hat{f}$ は $f$ の Fourier 変換と呼ばれる.) さらに, $\hat{f}(p)$ %
  は $p$ の連続函数になる. \qed
\end{question}

\begin{question}
  $f$ は $\R$ 上の連続函数であり, $f(x)$ と $xf(x)$ の %
  $(-\infty,\infty)$ における広義積分はともに絶対収束していると仮定す
  る. このとき, $f$ の Fourier 変換 $\hat{f}$ は微分可能であり, しかも,
  \[
    i \od{}{p} \hat{f}(p)
    =
    \int_{-\infty}^\infty x f(x) e^{-ipx} \,dx.
  \]
  が成立している. \qed
\end{question}

\begin{question}
  $f$ は $\R$ 上の $C^1$ 級函数であり, $f(x)$ と $f'(x)$ の %
  $(-\infty,\infty)$ における広義積分はともに絶対収束していると仮定す
  る. このとき, $f$ の Fourier 変換 $\hat{f}$ に関して, 
  \[
    p \hat{f}(p)
    =
    \int_{-\infty}^\infty \left( -i\od{}{x}f(x) \right) e^{-ipx} \,dx
  \qed
  \]
  および, $\lim\limits_{|p|\to\infty} \hat{f}(p) = 0$ が成立している. 
  \qed
\end{question}

\noindent ヒント: 部分積分を使う.

\begin{small}

\noindent 参考: このように, Fourier 変換は $x$ に関する微分作用素を %
$p$ に関する微分作用素に次の規則によって変換する: %
\(
  x \mapsto i \od{}{p},
  -i\od{}{x} \mapsto p.
\) %
また, $f(x)$ が $|x|\to\infty$ で $0$ に近付く速さが大きくなればなるほ
ど $\hat{f}(p)$ は滑らかになり, %
$f(x)$ が滑らかであればあるほど %
$\hat{f}(p)$ が $|p|\to\infty$ で $0$ に近付く速さは大きくなる.

\end{small}

\begin{question}
  ガンマ函数 %
  \(%
    \Gamma(s) = \int_0^\infty e^{-x}x^{s-1}\,dx
  \) %
  は $s>0$ における $C^\infty$ 級函数であることを示せ. 
  \qed
\end{question}

ヒント: 『解析概論』の pp.167--168.

\begin{question}
  この問題を解くために, %
  公式 $\int_0^\infty e^{-x^2} \,dx = \frac{\sqrt{\pi}}{2}$ を用いて良い. %
  積分変数 $x$ を $\sqrt{\alpha}x$ ($\alpha > 0$) で置換することによって, %
  \( %
    \int_0^\infty e^{-\alpha x^2}\,dx
    =
    \frac{1}{2} \sqrt{\frac{\pi}{\alpha}}
  \). %
  この式を $\alpha$ で微分し, $\alpha=1$ と置くことによって, 次が成立
  することがわかる:
  \[
    \int_0^\infty e^{-x^2} x^{2n} \,dx
    =
    \frac{1 \cdot 3 \cdots (2n-1)}{2^n} \frac{\sqrt{\pi}}{2}
    \qquad \text{for} \quad
    n = 0,1,2,\ldots.
  \]%
  さらに, 積分変数を $t = x^2$ と変換することによって, ガンマ函数に関
  する次の公式を導くことができる:
  \[
    \Gamma\left( n + \frac{1}{2} \right)
    =
    \int_0^\infty e^{-t}\, t^{n-1/2} \,dt
    =
    \frac{1 \cdot 3 \cdots (2n-1)}{2^n} \sqrt{\pi}
    \qquad \text{for} \quad
    n = 0,1,2,\ldots.
  \qed
  \]
\end{question}

\noindent ヒント: 『解析概論』の第48節の例5 (p.169).

\begin{question}
  この問題を解くために, %
  公式 $\int_{-\infty}^\infty e^{-x^2} \,dx = \sqrt{\pi}$ を用いて良い. %
  $\alpha\in\R$ に対して, 次が成立することを示せ:
  \[
    \int_{-\infty}^\infty e^{- x^2 + 2 \alpha x} \,dx
    =
    e^{- \alpha^2} \sqrt{\pi}.
  \]
  両辺を $\alpha$ で微分して, $\alpha = 0$ と置くことによって, 
  \[
    \int_{-\infty}^\infty e^{-x^2} x^{2n} \,dx
    =
    \frac{1 \cdot 3 \cdots (2n-1)}{2^n} \sqrt{\pi}
    \qquad \text{for} \quad
    n = 0,1,2,\ldots
  \]%
  が成立することがわかる. \qed
\end{question}

\begin{question}
  次の公式が成立することを示せ:
  \[
    \int_{-\infty}^\infty e^{-x^2}|x|^{2n+1} \,dx = n!
    \qquad \text{for} \quad
    n = 0,1,2,\ldots.
  \qed
  \]
\end{question}

\begin{question}
  $p\in\R$ に対して,
  \( \displaystyle%
    \int_0^\infty e^{-x^2} \cos px \,dx
    =
    \frac{\sqrt{\pi}}{2} e^{-p^2/4}
  \). %
  \qed
\end{question}

\noindent ヒント: 『解析概論』の第48節の例6 (p.170). %
左辺を $F(p)$ と置く. 積分記号下での微分と部分積分によって, $F$ は微分
方程式 $F'(p) = - \frac{p}{2} F(p)$ を満たしていることがわかる. この方
程式を解くと, $F(p) = c e^{-p^2/4}$. 定数 $c$ は $F(0)$ を計算すれば求
められる.

\begin{question}
  公式 $\int \frac{dx}{x^2+1} = \arctan x$ を使うと, %
  $\int_{-\infty}^\infty \frac{dx}{x^2+1} = \pi$ となることがわかる. %
  積分変数 $x$ を $x/\sqrt{\alpha}$ で置換し, $\alpha$ に関する積分記
  号下の微分を使って, 次が成立することを示せ:
  \[
    \int_{-\infty}^\infty \frac{dt}{(x^2 + 1)^{n+1}}
    =
    \frac{1}{2} \cdot \frac{3}{4} \cdots \frac{2n-1}{2n} \cdot \pi
    \qquad \text{for} \quad
    n = 0,1,2,\ldots.
  \qed
  \]
\end{question}

\noindent ヒント: 『解析概論』の第4章の練習問題(8) (p.199).

%%%%%%%%%%%%%%%%%%%%%%%%%%%%%%%%%%%%%%%%%%%%%%%%%%%%%%%%%%%%%%%%%%%%%%%%%%%

%\section{最後に}
%
%\begin{question}
%  この演習の感想を書け. \qed
%\end{question}

%%%%%%%%%%%%%%%%%%%%%%%%%%%%%%%%%%%%%%%%%%%%%%%%%%%%%%%%%%%%%%%%%%%%%%%%%%%
\end{document}
%%%%%%%%%%%%%%%%%%%%%%%%%%%%%%%%%%%%%%%%%%%%%%%%%%%%%%%%%%%%%%%%%%%%%%%%%%%
