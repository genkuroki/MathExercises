%%%%%%%%%%%%%%%%%%%%%%%%%%%%%%%%%%%%%%%%%%%%%%%%%%%%%%%%%%%%%%%%%%%%%%%%%%%%
%\def\STUDENT{} % \def すると計算問題の解答を印刷しなくなる.
%%%%%%%%%%%%%%%%%%%%%%%%%%%%%%%%%%%%%%%%%%%%%%%%%%%%%%%%%%%%%%%%%%%%%%%%%%%%
\documentclass[12pt,twoside]{jarticle}
%\documentclass[12pt]{jarticle}
\usepackage{amsmath,amssymb,amscd}
\usepackage{eepic}
\usepackage{enshu}
\newcommand\qstar[1]{}
%\usepackage{showkeys}
\allowdisplaybreaks
%%%%%%%%%%%%%%%%%%%%%%%%%%%%%%%%%%%%%%%%%%%%%%%%%%%%%%%%%%%%%%%%%%%%%%%%%%%%
\newlabel{q:kappa(t)-tau(t)}{{26}{10}}
\newlabel{q:Fourier-coeff}{{65}{23}}
%%%%%%%%%%%%%%%%%%%%%%%%%%%%%%%%%%%%%%%%%%%%%%%%%%%%%%%%%%%%%%%%%%%%%%%%%%%%
\setcounter{page}{25}      % この数から始まる
\setcounter{section}{4}    % この数の次から始まる
\setcounter{theorem}{0}    % この数の次から始まる
\setcounter{question}{67}  % この数の次から始まる
\setcounter{footnote}{0}   % この数の次から始まる
%%%%%%%%%%%%%%%%%%%%%%%%%%%%%%%%%%%%%%%%%%%%%%%%%%%%%%%%%%%%%%%%%%%%%%%%%%%%
\ifx\STUDENT\undefined
%
% 教師専用
%
\newcommand\commentout[1]{#1}
%%%%%%%%%%%%%%%%%%%%%%%%%%%%%%%%%%%%%%%%%%%%%%%%%%%%%%%%%%%%%%%%%%%%%%%%%%%%
\else
%%%%%%%%%%%%%%%%%%%%%%%%%%%%%%%%%%%%%%%%%%%%%%%%%%%%%%%%%%%%%%%%%%%%%%%%%%%%
%
% 生徒専用
%
\newcommand\commentout[1]{}
%%%%%%%%%%%%%%%%%%%%%%%%%%%%%%%%%%%%%%%%%%%%%%%%%%%%%%%%%%%%%%%%%%%%%%%%%%%%
\fi
%%%%%%%%%%%%%%%%%%%%%%%%%%%%%%%%%%%%%%%%%%%%%%%%%%%%%%%%%%%%%%%%%%%%%%%%%%%%
\begin{document}
%%%%%%%%%%%%%%%%%%%%%%%%%%%%%%%%%%%%%%%%%%%%%%%%%%%%%%%%%%%%%%%%%%%%%%%%%%%%
%\title{\bf 幾何学序論B演習
%  \ifx\STUDENT\undefined\\{\normalsize 教師用\quad(計算問題の略解付き)}\fi}
%\author{黒木 玄 \quad (東北大学大学院理学研究科数学専攻)}
%\date{2005年10月4日(火)}
%\maketitle
%%%%%%%%%%%%%%%%%%%%%%%%%%%%%%%%%%%%%%%%%%%%%%%%%%%%%%%%%%%%%%%%%%%%%%%%%%%%
\noindent
{\Large\bf 幾何学序論B演習}
\hfill
{\large 黒木玄}
\qquad
2005年11月29日(火)
\commentout{\quad (教師用)}
%%%%%%%%%%%%%%%%%%%%%%%%%%%%%%%%%%%%%%%%%%%%%%%%%%%%%%%%%%%%%%%%%%%%%%%%%%%%
\tableofcontents
%%%%%%%%%%%%%%%%%%%%%%%%%%%%%%%%%%%%%%%%%%%%%%%%%%%%%%%%%%%%%%%%%%%%%%%%%%%%

\section*{先週渡したプリントの訂正}

\qref{q:Fourier-coeff} の $\d ta_{m,n}$ は正しくは $\delta_{m,n}$ である.
最後のヒントの $\int_0^\infty$ は正しくは $\int_0^{2\pi}$ である.

\section*{問題 \qref{q:kappa(t)-tau(t)} の略解}

\begin{proof}[略解]
 弧長パラメータを $s$ と書き, $t$ の函数 $f$ を $s$ の函数ともみなし, 
 $s$ による微分を $f'$ と書くことにする. 
 弧長パラメータの定義より $f' = \dot f/|\dot q|$ である.

 方針: $e_1,q'',\kappa,e_2,e_3$ を順次 $\dot q$, $\ddot q$ で
 表わし, $\tau = e_3\cdot e_2'$ を計算する. 
 そのとき $\dot q$, $\ddot q$ が $e_3$ と直交することを用いて
 計算を簡略化する.

 まず最初に $e_1=q'$, $d|\dot q|/dt$, $e_1'=q''$ を $\dot q$, $\ddot q$ 
 で表わそう:
 \begin{align*}
  &
  e_1 = q' = \frac{\dot q}{|\dot q|},
  \\ &
  \frac{d|\dot q|}{dt}
  = \frac{d}{dt}\sqrt{\dot q\cdot \dot q}
  = \frac{2\dot q\cdot\ddot q}{2\sqrt{\dot q\cdot \dot q}}
  = \frac{\dot q\cdot\ddot q}{|\dot q|},
  \\ &
  e_1'=q''=\frac{1}{|\dot q|}\frac{d}{dt}\frac{\dot q}{|\dot q|}
  = \frac{\ddot q}{|\dot q|^2} 
  - \frac{(\dot q\cdot\ddot q)\dot q}{|\dot q|^4}
  = \frac{|\dot q|^2\ddot q - (\dot q\cdot\ddot q)\dot q}{|\dot q|^4}.
 \end{align*}
 よって $\kappa^2 = |q''|^2$ は次の形になる:
 \begin{align*}
  \kappa^2=|q''|^2
  &
  = \frac{|\dot q|^4|\ddot q|^2 
  - 2|\dot q|^2(\dot q\cdot \ddot q)^2 
  + |\dot q|^2(\dot q\cdot \ddot q)^2}{|\dot q|^8}
%  \\ &
  = \frac{|\dot q|^2|\ddot q|^2-(\dot q\cdot \ddot q)^2}{|\dot q|^6}
  = \frac{|\dot q\times \ddot q|^2}{|\dot q|^6}.
 \end{align*}
 ここで公式 $|a\times b|^2=|a|^2|b|^2\sin^2\theta=|a|^2|b|^2-(a\cdot b)^2$
 ($a,b\in\R^3$, $\theta$ は $a$ と $b$ のあいだの角度) を使った. よって
 \begin{align*}
  &
  \kappa = |q''| = \frac{|\dot q\times \ddot q|}{|\dot q|^3},
  \qquad
%  \\ &
  e_2 = \frac{q''}{|q''|} 
  = \frac{\ddot q}{\kappa|\dot q|^2} 
  - \frac{(\dot q\cdot\ddot q)\dot q}{\kappa|\dot q|^4},
  \qquad
%  \\ &
  e_3=e_1\times e_2
  = \frac{\dot q\times \ddot q}{\kappa|\dot q|^3}.
 \end{align*}
 最後に $\tau=e_3\cdot e_2'$ を求めよう:
 \begin{align*}
  &
  \tau = e_3\cdot e_2' = \frac{1}{|\dot q|}e_3\cdot\dot e_2
  = 
  \frac{1}{|\dot q|} 
  \frac{\dot q\times \ddot q}{\kappa|\dot q|^3}
  \cdot
  \frac{\dddot q}{\kappa|\dot q|^2}
  = \frac{(\dot q\times \ddot q)\cdot \dddot q}{|\dot q\times \ddot q|^2}
  = \frac{\det[\dot q,\ddot q,\dddot q]}{|\dot q\times \ddot q|^2}.
 \end{align*}
 ここで3番目の等号で $\dot q$, $\ddot q$ と $e_3$ が直交することを使い, 
 最後の等号でベクトル解析の
 公式 $(a\times b)\cdot c=\det[a,b,c]$ ($a,b,c\in\R^3$) を使った.
 \qed
\end{proof}

%%%%%%%%%%%%%%%%%%%%%%%%%%%%%%%%%%%%%%%%%%%%%%%%%%%%%%%%%%%%%%%%%%%%%%%%%%%%
\end{document}
%%%%%%%%%%%%%%%%%%%%%%%%%%%%%%%%%%%%%%%%%%%%%%%%%%%%%%%%%%%%%%%%%%%%%%%%%%%%
