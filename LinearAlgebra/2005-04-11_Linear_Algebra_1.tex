%%%%%%%%%%%%%%%%%%%%%%%%%%%%%%%%%%%%%%%%%%%%%%%%%%%%%%%%%%%%%%%%%%%%%%%%%%%%
%\def\STUDENT{} % \def すると計算問題の解答を印刷しなくなる.
%%%%%%%%%%%%%%%%%%%%%%%%%%%%%%%%%%%%%%%%%%%%%%%%%%%%%%%%%%%%%%%%%%%%%%%%%%%%
%
% 線形代数学演習---行列の標準形
% 
% 黒木 玄 (東北大学理学部数学教室, kuroki@math.tohoku.ac.jp)
%
% この演習問題集は2005年度における東北大学理学部数学科2年生前期の
% 代数学序論B演習のために作成されました. 
%
%%%%%%%%%%%%%%%%%%%%%%%%%%%%%%%%%%%%%%%%%%%%%%%%%%%%%%%%%%%%%%%%%%%%%%%%%%%%
\documentclass[12pt,twoside]{jarticle}
%\documentclass[12pt]{jarticle}
\usepackage{amsmath,amssymb,amscd}
\usepackage{enshu}
%\usepackage{showkeys}
\allowdisplaybreaks
%%%%%%%%%%%%%%%%%%%%%%%%%%%%%%%%%%%%%%%%%%%%%%%%%%%%%%%%%%%%%%%%%%%%%%%%%%%%
\ifx\STUDENT\undefined
%
% 教師専用
%
\newcommand\commentout[1]{#1}
\setcounter{page}{1}       % この数から始まる
\setcounter{section}{0}    % この数の次から始まる
\setcounter{theorem}{0}    % この数の次から始まる
\setcounter{question}{0}   % この数の次から始まる
%%%%%%%%%%%%%%%%%%%%%%%%%%%%%%%%%%%%%%%%%%%%%%%%%%%%%%%%%%%%%%%%%%%%%%%%%%%%
\else
%%%%%%%%%%%%%%%%%%%%%%%%%%%%%%%%%%%%%%%%%%%%%%%%%%%%%%%%%%%%%%%%%%%%%%%%%%%%
%
% 生徒専用
%
\newcommand\commentout[1]{}
\setcounter{page}{1}       % この数から始まる
\setcounter{section}{0}    % この数の次から始まる
\setcounter{theorem}{0}    % この数の次から始まる
\setcounter{question}{0}   % この数の次から始まる
\setcounter{footnote}{0}   % この数の次から始まる
%%%%%%%%%%%%%%%%%%%%%%%%%%%%%%%%%%%%%%%%%%%%%%%%%%%%%%%%%%%%%%%%%%%%%%%%%%%%
\fi
%%%%%%%%%%%%%%%%%%%%%%%%%%%%%%%%%%%%%%%%%%%%%%%%%%%%%%%%%%%%%%%%%%%%%%%%%%%%
\begin{document}
%%%%%%%%%%%%%%%%%%%%%%%%%%%%%%%%%%%%%%%%%%%%%%%%%%%%%%%%%%%%%%%%%%%%%%%%%%%%

\title{\bf 線形代数学演習---行列の標準形
%  \thanks{この演習問題集は2005年度における東北大学理学部数学科2年生前期の
%    代数学序論B演習のために作成された.}
%  \ifx\STUDENT\undefined\\{\normalsize 教師用\quad(計算問題の略解付き)}\fi}
%  \ifx\STUDENT\undefined\\{\normalsize 計算問題の略解付き}\fi}
  }

\author{黒木 玄 \quad (東北大学大学院理学研究科数学専攻)}

%\date{最終更新2003年11月21日 \quad (作成2005年4月11日)}
\date{2004年4月11日}

\maketitle
%%%%%%%%%%%%%%%%%%%%%%%%%%%%%%%%%%%%%%%%%%%%%%%%%%%%%%%%%%%%%%%%%%%%%%%%%%%%

\tableofcontents

%%%%%%%%%%%%%%%%%%%%%%%%%%%%%%%%%%%%%%%%%%%%%%%%%%%%%%%%%%%%%%%%%%%%%%%%%%%%

%%%%%%%%%%%%%%%%%%%%%%%%%%%%%%%%%%%%%%%%%%%%%%%%%%%%%%%%%%%%%%%%%%%%%%%%%%%
\section{論理と集合}
%%%%%%%%%%%%%%%%%%%%%%%%%%%%%%%%%%%%%%%%%%%%%%%%%%%%%%%%%%%%%%%%%%%%%%%%%%%

問題に誤りがある場合には訂正してから解くこと.

%%%%%%%%%%%%%%%%%%%%%%%%%%%%%%%%%%%%%%%%%%%%%%%%%%%%%%%%%%%%%%%%%%%%%%%%%%%%

$n$ 次元実ベクトル空間 $\R^n$ の線形部分空間 $W$ と $v\in\R^n$ に
対して $\R^n$ の部分集合 $v+W$ を次のように定める:
\begin{equation*}
  v + W := \{\, v+w \mid w\in W \,\}.
\end{equation*}

\begin{question}[10点]
  $u,v \in\R^n$ に対して以下の条件は互いに同値である:
  \begin{itemize}
  \item[(a)] $u+W = v+W$,
  \item[(b)] $u \in v+W$,
  \item[(c)] $u-v \in W$.
    \qed
  \end{itemize}
\end{question}

%%%%%%%%%%%%%%%%%%%%%%%%%%%%%%%%%%%%%%%%%%%%%%%%%%%%%%%%%%%%%%%%%%%%%%%%%%%%

\begin{question}
以下の文章の否定文を書け:
\begin{enumerate}
\renewcommand{\labelenumi}{(例)}
\item この演習は面白い. \quad$\longrightarrow$\quad この演習は面白くない. 
\end{enumerate}
\begin{enumerate}
\renewcommand{\labelenumi}{(\arabic{enumi})}
\item AならばBである. (2点)
\item AとBの両方が同時に成立することはないが, どちらか片方は成立している. (2点)
\item 三毛猫じゃない猫もいる. (2点)
\item 大学のすべての講義は面白い. (2点)
\item 三毛猫じゃない猫も結構たくさんいる. (5点)
\item 大学のほとんどすべての講義はつまらない. (5点) 
\end{enumerate}
ただし「AならばBである」の否定を「「AならばB」が成立しない」と
するような解答は不可であるとする.
最後の2問についてはどうしてそのような解答になったかを
できるだけ詳しく説明すること.
\qed
\end{question}

%%%%%%%%%%%%%%%%%%%%%%%%%%%%%%%%%%%%%%%%%%%%%%%%%%%%%%%%%%%%%%%%%%%%%%%%%%%%

この演習では集合 $A$ が集合 $B$ の部分集合であることを $A\subset B$ と
書き, $A$ が $B$ の部分集合でかつ $A$ と $B$ が等しくないとき $A\subsetneqq B$ 
と書くことにする.

集合間の写像 $f:X\to Y$ と $A\subset X$, $B\subset Y$ に対して, %
$A$ の $f$ による像 $f(A)$ と $B$ の $f$ による逆像 $f^{-1}(B)$ を次の
ように定める:
\begin{align*}
  &
  f(A)
  = \{\, f(x) \mid x \in A\,\}
  = \{\, y\in Y\mid \text{ある $x\in A$ で $y=f(x)$ となるものが存在する}\,\},
  \\ &
  f^{-1}(B) = \{\, x\in X\mid f(x)\in B\,\}.
\end{align*}

\begin{question}[5点]
  $f(f^{-1}(B))=B\cap f(X)$. \qed
\end{question}

\begin{proof}[参考]
  $f^{-1}(f(A))$ についてはレポート問題を見よ. \qed
\end{proof}

\begin{question}[5点]
  $A,A'\subset X$ に対して $f(A\cup A')=f(A)\cup f(A')$. \qed
\end{question}

\begin{proof}[参考]
  $f(A\cap A')$ についてはレポート問題を見よ. \qed
\end{proof}

\begin{question}[5点]
  $B,B'\subset Y$ に対して %  
  $f^{-1}(B\cap B') = f^{-1}(B)\cap f^{-1}(B')$ かつ \\%
  $f^{-1}(B\cup B') = f^{-1}(B)\cup f^{-1}(B')$.
  \qed
\end{question}

%%%%%%%%%%%%%%%%%%%%%%%%%%%%%%%%%%%%%%%%%%%%%%%%%%%%%%%%%%%%%%%%%%%%%%%%%%%%
%\begin{question}
%  \( \{ 0, 1 \} = \{ 0,\, 0,\, 1 \} \) を証明せよ.
%  \qed
%\end{question}
%
%\begin{proof}[ヒント]
%集合 $A$ と集合 $B$ が等しいとは, 条件
%\( x \in A \)
%と条件
%\( x \in B \)
%が同値になることであると定義される. 
%集合 $\{a,b\}$ は次によって定義される:
%\[
%  x \in \{a,b\}
%  \quad \Longleftrightarrow \quad
%  x = a\quad \text{または}\quad x = b.
%\qed
%\]
%\end{proof}

%%%%%%%%%%%%%%%%%%%%%%%%%%%%%%%%%%%%%%%%%%%%%%%%%%%%%%%%%%%%%%%%%%%%%%%%%%%

集合 $A$, $B$ に対して直積集合 $A\times B$ と羃集合 $B^A$ を次のように定義する:
\begin{align*}
  &
  A\times B := \{\,(x,y)\mid x\in X,\,y\in Y \,\},
  \\ &
  B^A := \{\, f \mid \text{$f$ は $A$ から $B$ への写像である} \,\}.
\end{align*}

\begin{question}[10点]
二つの有限集合 $A$, $B$ に対して次が成立する:
\[
  \text{(1)}\quad |A \times B| = |A| \times |B|,
  \qquad
  \text{(2)}\quad |B^A| = |B|^{|A|}.
\qed
\]
\end{question}

%%%%%%%%%%%%%%%%%%%%%%%%%%%%%%%%%%%%%%%%%%%%%%%%%%%%%%%%%%%%%%%%%%%%%%%%%%%

%\begin{question}
%$X$ は任意の集合とし, 集合 $\Two$ を $\Two:=\{0,1\}$ と定義すると
%次が成立する:
%\[
%  |\Power(X)| = |\Two^X|.
%  \qed
%\]
%\end{question}
%
%\begin{proof}[解説]
%二つの集合 $A$, $B$ に対して, \( |A| = |B| \) であるとは, $A$ から $B$
%への写像 $f$ で逆写像を持つものが存在することであると定義される. 
%\qed
%\end{proof}

%%%%%%%%%%%%%%%%%%%%%%%%%%%%%%%%%%%%%%%%%%%%%%%%%%%%%%%%%%%%%%%%%%%%%%%%%%%

\begin{question}[10点]
我々は, 条件 $\text{P}(x)$ を満たす $x$ 全体のなす集合のことを次のよう
に書くのであった:
\[
  \{\, x \mid \text{P}(x) \,\}.
\]
この記号法は便利なのであるが, この記号法を無制限に用いると, 矛盾を簡単
に導けることが知られている. 例えば, 集合 $S$ を次の様に定義する:
\[
  S := \{\, x \mid x \notin x \,\}.
\]
このとき, $S \in S$ と仮定しても, $S \notin S$ と仮定しても, 矛盾が導
かれることを説明せよ. (ヒント: 矛盾とはある条件 Q とその否定 not Q が
同時に成立することが示された状態のことである. )
\qed
\end{question}

\begin{proof}[解説]
(1) この paradox を Russel の逆理と呼ぶ. この逆理は今世紀の始め(1902年
頃)に発見された. 逆理 (paradox) とは一見不合理もしくは矛盾しているよう
で実は正しい説のことである. 1930年代(すでに大昔)に, このような矛盾が生
じない(ことがほとんど確実であると思われる)公理的な集合論が整備されてい
る. 通常の公理系において上の $S$ は集合全体の集りに等しくなるので, 
Russel の逆理は, 集合全体の集り $S$ は集合であると考えてはいけないこと
を表わしていると思うこともできる. 
\par\noindent
(2) Russel の逆理の構造は Cantor の対角線論法の構造と密接に関係している. 
\par\noindent
(3) Russel 型の逆理は
\( \{\, x \mid \text{P}(x) \,\} \)
という記号法を以下のような場合に制限して用いる限り生じない:
\[
  B = \{ x \mid x \in A\ \text{and}\ \text{P}(x) \,\}.
\]
ここで, $A$ は任意の集合である. この集合 $B$ は以下の様に略記されるの
が普通であり, この略記法はよく使われる:
\[
  B = \{ x \in A \mid \text{P}(x) \,\}.
\]
なお, この $B$ は $A$ の部分集合になり, $A$ の任意の部分集合はこの形に
表わすことができる. 
\qed 
\end{proof}

%%%%%%%%%%%%%%%%%%%%%%%%%%%%%%%%%%%%%%%%%%%%%%%%%%%%%%%%%%%%%%%%%%%%%%%%%%%

\begin{question}[10点]
$X$, $Y$ は任意の集合とし, $A \subset X$, $B \subset Y$ であるとす
る. このとき, 以下が成立する:

\par\noindent
(1) 自然に
\( A \times B \subset X \times Y \)
とみなせる. 

\par\noindent
(2) 補集合達を, 
\( A^c = X - A \),
\( B^c = Y - B \),
\( (A \times B)^c = (X \times Y) - (A \times B) \)
と書くと, 
\[
  (A \times B)^c = (A^c \times Y) \cup (X \times B^c).
  \qed
\]
\end{question}

\begin{proof}[解説]
補集合を表わす記号には以下のようなものがある:
\[
    A^c = X - A = X \setminus A = \{\,x \in X \mid x \notin A\,\}.
\]
記号 $\bar A$ は閉包を表わすために使われることが多い. 
\qed
\end{proof}

%%%%%%%%%%%%%%%%%%%%

%\begin{question}
%$X$ は任意の集合とし, その部分集合全体の集合を $\Power(X)$ と書く. 
%このとき, 以下が成立することを示せ:
%
%\par\noindent
%(1) \qquad \( |X| \le |\Power(X)| \).
%
%\par\noindent
%(2) \qquad \( |X|  <  |\Power(X)| \).
%\qed
%\end{question}
%
%
%\begin{proof}[ヒント]
%(1) は簡単. (2)は Cantor の対角線論法を使う. 
%すなわち $|\Power(X)|=|X|$ を仮定して矛盾を導く.
%\qed  
%\end{proof}
%
%\begin{proof}[解説]
%Cantor の対角線論法によって矛盾を導く方法は, Russel の paradox (問題 
%[7]) において矛盾が現れる仕組と全く同じ構造を持っている. 
%\end{proof}

%%%%%%%%%%%%%%%%%%%%%%%%%%%%%%%%%%%%%%%%%%%%%%%%%%%%%%%%%%%%%%%%%%%%%%%%%%%

%\begin{question}
%実数全体の集合 $\R$ (幾何学的には実直線)と $\R^2 = \R \times \R$ (幾何
%学的には実平面)の間に一対一対応が存在することを証明せよ. 
%\qed
%\end{question}

%\begin{proof}[解説]
%(1) すなわち, 集合論的には実直線と実平面は同型であることをこの問題は主
%張している. このことは, 直観に反するようであるが論理的には正しい. それ
%では, 実直線と実平面を区別するためにはどのようにしたら良いのであろうか?
%この疑問は位相 (topology) の概念を学ぶことによって解決するであろう. 
%
%\par\noindent
%(2) 位相を考えたとしても, 面白いことに, 次が成立することを証明できる:
%$\R$ から $\R^2$ への連続な全射が存在する. Peano による例が有名である. 
%実直線から実平面への写像は, 平面の上に曲線を描くが, Peano の例に対する
%曲線は Peano 曲線と呼ばれている. 
%\end{proof}

%%%%%%%%%%%%%%%%%%%%%%%%%%%%%%%%%%%%%%%%%%%%%%%%%%%%%%%%%%%%%%%%%%%%%%%%%%%

\bigskip

この演習で以下の問題を必ずしも解く必然性はないが, 
ここで解いておけば後でより進んだ代数学を勉強するときに役に立つかもしれない.

\par\medskip\noindent
写像 $f\colon X\to Y$ と写像 $g\colon Y\to Z$ の
合成 $g\circ f\colon X\to Z$ を次に
よって定める:
\[
  (g\circ f)(x) := g(f(x))
  \qquad (x\in X).
\]

%%%%%%%%%%%%%%%%%%%%%%%%%%%%%%%%%%%%%%%%%%%%%%%%%%%%%%%%%%%%%%%%%%%%%%%%%%%

\begin{question}[10点]
$X$ と $Y$ は集合であるとし, 
写像 $p_X\colon X\times Y\to X$, $p_Y\colon X\times Y\to Y$ 
を次の様に定める:
\[
  p_X(x,y) := x, \qquad p_Y(x,y):= y
  \qquad ((x,y)\in X\times Y).
\]
このとき, 任意の集合 $S$ 
と写像 $s_X\colon S\to X$, $s_Y\colon S\to Y$ に対して, 
次を満たす写像 $s\colon S\to X\times Y$ が唯一存在する:
\[
  p_X \circ s = s_X, \qquad  p_Y \circ s = s_Y.
  \qed
\]
\end{question}


%%%%%%%%%%%%%%%%%%%%%%%%%%%%%%%%%%%%%%%%%%%%%%%%%%%%%%%%%%%%%%%%%%%%%%%%%%%

\begin{question}[10点]
\label{q:9}
$X$ と $Y$ は互いに交わらない集合であるとする. 
このとき, $X\cup Y$ は $X$, $Y$ の直和 (direct sum, disjoint union)
であると言う. 
写像 $i_X\colon X\to X\cup Y$, $i_Y\colon Y\to X\cup Y$ 
を次の様に定める:
\[
  i_X(x) := x, \qquad i_Y(y):= y
  \qquad (x\in X,\, y\in Y)
\]
このとき, 任意の集合 $S$ 
と写像 $s_X\colon X\to S$, $s_Y\colon Y\to S$ に対して, 
次を満たす写像 $s\colon X\cup Y\to S$ が唯一存在する:
\[
  s \circ i_X = s_X, \qquad  s \circ i_Y = s_Y.
  \qed
\]
\end{question}

%%%%%%%%%%%%%%%%%%%%%%%%%%%%%%%%%%%%%%%%%%%%%%%%%%%%%%%%%%%%%%%%%%%%%%%%%%%

\begin{question}[10点]
\label{q:10}
$X$ と $Y$ は集合であるとし, 二つの写像 $f,g\colon X\to Y$ を考える. 
集合 $K$ と写像 $i\colon K\to X$ を次のように定める:
\[
  K := \{\,x\in X \mid f(x)=g(x)\,\},
  \qquad
  i(x) := x \quad (x\in K).
\]
このとき, $f\circ i = g\circ i$ が成立する. 
さらに, 任意の集合 $S$ と任意の写像 $s\colon S\to X$ に対して, 
$f\circ s = g\circ s$ ならば, 
次を満たす写像 $j\colon S\to K$ が唯一存在する: $i\circ j = s$.
\qed
\end{question}


\begin{proof}[解説]
この問題の $K$ は $f$ と $g$ の差核 (difference kernel, equalizer) と
呼ばれている. \qed
\end{proof}

%%%%%%%%%%%%%%%%%%%%%%%%%%%%%%%%%%%%%%%%%%%%%%%%%%%%%%%%%%%%%%%%%%%%%%%%%%%

%\pagebreak

\begin{question}[10点]
$X$, $Y$, $Z$ は集合であるとし, 
写像 $f_X\colon X\to Z$, $f_Y\colon Y\to Z$ を考える. 
集合 $P$ と写像 $p_X\colon P\to X$, $p_Y\colon P\to Y$ を
次のように定める:
\begin{align*}
  & P := \{\,(x,y)\in X\times Y \mid f_X(x)=f_Y(y)\,\}, \\
  & p_X(x,y) := x, \quad p_Y(x,y) := y \quad ((x,y)\in P).
\end{align*}
このとき, $f_X\circ p_X = f_Y\circ p_Y$ が成立する. 
さらに, 任意の集合 $S$ と
任意の写像 $s_X\colon S\to X$, $s_Y\colon S\to Y$ に対して, 
$f_X\circ s_X = f_Y\circ s_Y$ ならば, 
次を満たす写像 $s\colon S\to P$ が唯一存在する:
\[
  p_X\circ s = s_X, \qquad p_Y\circ s = s_Y.
  \qed
\]
\end{question}

\begin{proof}[参考]
この問題の $P$ は $f_X$ と $f_Y$ のファイバー積 (fiber product) と呼ば
れている.
\qed
\end{proof}

\begin{proof}[参考]
(1) 以上の4つの問題は, 直積・直和・差核・ファイバー積な
どの 普遍性 (universality) を証明せよという問題である. 数学における多
くの自然な定義は universality による特徴付けになっている.

\par\noindent
(2) \qref{q:9}, \qref{q:10} の双対性に注意せよ. 
直積と直和では写像の向きが逆になるだけで同様な命題が成立している. 
もちろん, 差核やファイバー積の双対概念も存在する.
\qed
\end{proof}

%%%%%%%%%%%%%%%%%%%%%%%%%%%%%%%%%%%%%%%%%%%%%%%%%%%%%%%%%%%%%%%%%%%%%%%%%%%

\begin{question}[10点]
写像 $f \colon X \to Y$ に対して, 以下の2条件は互いに同値である:
\par\medskip\noindent
(a) $f$ は単射である.
\par\medskip\noindent
(b) 任意の集合 $A$ と任意の2つの写像 $g,h \colon A \to X$ に対して,
$f \circ g = f \circ h$ ならば $g = h$ である.
\qed
\end{question}

%%%%%%%%%%%%%%%%%%%%%%%%%%%%%%%%%%%%%%%%%%%%%%%%%%%%%%%%%%%%%%%%%%%%%%%%%%%

\begin{question}[10点]
写像 $f \colon X \to Y$ に対して, 以下の2条件は互いに同値である:
\par\medskip\noindent
(a) $f$ は全射である.
\par\medskip\noindent
(b) 任意の集合 $B$ と任意の2つの写像 $g,h \colon Y \to B$ に対して,
$g \circ f = h \circ f$ ならば $g = h$ である.
\qed
\end{question}

%%%%%%%%%%%%%%%%%%%%%%%%%%%%%%%%%%%%%%%%%%%%%%%%%%%%%%%%%%%%%%%%%%%%%%%%%%%

%\begin{thebibliography}{ABC}
%
%\bibitem[佐武]{satake} 佐武一郎: 線型代数学, 裳華房数学選書 1, 324頁.
%
%\bibitem[齋藤]{saito} 齋藤正彦: 線型代数入門, 東京大学出版会基礎数学 
%1, 278頁.
%
%\bibitem[堀田]{hotta} 堀田良之: 加群十話 --- 代数学入門, 朝倉書店, すうがく
%ぶっくす 3, 186頁.
%
%\bibitem[志賀]{shiga}
%志賀浩二: 集合への30講, 朝倉書店 数学30講シリーズ 3, 187頁.
%
%\end{thebibliography}

%%%%%%%%%%%%%%%%%%%%%%%%%%%%%%%%%%%%%%%%%%%%%%%%%%%%%%%%%%%%%%%%%%%%%%%%%%%
\end{document}
%%%%%%%%%%%%%%%%%%%%%%%%%%%%%%%%%%%%%%%%%%%%%%%%%%%%%%%%%%%%%%%%%%%%%%%%%%%
